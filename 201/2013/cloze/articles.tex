People are, on the whole, poor at considering background information when making individual decisions. At first glance this might seem like a strength that \uline{~~1~~} the ability to make judgments which are unbiased by \uline{~~2~~} factors. But Dr Uri Simonsohn speculated that an inability to consider the big \uline{~~3~~} was leading decision-makers to be biased by the daily samples of information they were working with. \uline{~~4~~}, he theorised that a judge \uline{~~5~~} of appearing too soft \uline{~~6~~} crime might be more likely to send someone to prison \uline{~~7~~} he had already sentenced five or six other defendants only to forced community service on that day.


To \uline{~~8~~} this idea, he turned to the university-admissions process. In theory, the \uline{~~9~~} of an applicant should not depend on the few others \uline{~~10~~} randomly for interview during the same day, but Dr Simonsohn suspected the truth was \uline{~~11~~}.


He studied the results of 9,323 MBA interviews \uline{~~12~~} by 31 admissions officers. The interviewers had \uline{~~13~~} applicants on a scale of one to five. This scale \uline{~~14~~} numerous factors into consideration. The scores were \uline{~~15~~} used in conjunction with an applicant's score on the Graduate Management Admission Test, or GMAT, a standardised exam which is \uline{~~16~~} out of 800 points, to make a  decision on whether to accept him or her.


Dr Simonsohn found if the score of the previous candidate in a daily series of interviewees was 0.75 points or more higher than that of the one \uline{~~17~~} that, then  the score for the next applicant would \uline{~~18~~} by an average of 0.075 points. This  might sound small, but to \uline{~~19~~} the effects of such a decrease a candidate would  need 30 more GMAT points than would otherwise have been \uline{~~20~~}.