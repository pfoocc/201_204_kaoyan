Up until a few decades ago, our visions of the future were largely – though by no means uniformly – glowingly positive. Science and technology would cure all the ills of humanity, leading to lives of fulfilment and opportunity for all.


Now utopia has grown unfashionable, as we have gained a deeper appreciation of the range of threats facing us, from asteroid strike to epidemic flu and to climate change. You might even be tempted to assume that humanity has little future to look forward to.


But such gloominess is misplaced. The fossil record shows that many species have endured for millions of years – so why shouldn't we? Take a broader look at our species' place in the universe, and it becomes clear that we have an excellent chance of surviving for tens, if not hundreds, of thousands of years. Look up Homo sapiens in the ``Red List'' of threatened species of the International Union for the Conservation of Nature (IUCN) and you will read: ``Listed as Least Concern as the species is very widely distributed, adaptable, currently increasing, and there are no major threats resulting in an overall population decline.''


So what does our deep future hold? A growing number of researchers and organisations are now thinking seriously about that question. For example, the Long Now Foundation has as its flagship project a mechanical clock that is designed to still be marking time thousands of years hence.


Perhaps willfully, it may be easier to think about such lengthy timescales than about the more immediate future. The potential evolution of today's technology, and its social consequences, is dazzlingly complicated, and it's perhaps best left to science fiction writers and futurologists to explore the many possibilities we can envisage. That's one reason why we have launched Arc, a new publication dedicated to the near future.


But take a longer view and there is a surprising amount that we can say with considerable assurance. As so often, the past holds the key to the future: we have now identified enough of the long-term patterns shaping the history of the planet, and our species, to make evidence-based forecasts about the situations in which our descendants will find themselves.


This long perspective makes the pessimistic view of our prospects seem more likely to be a passing fad. To be sure, the future is not all rosy. But we are now knowledgeable enough to reduce many of the risks that threatened the existence of earlier humans, and to improve the lot of those to come.