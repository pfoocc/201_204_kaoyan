On a five to three vote, the Supreme Court knocked out much of Arizona's immigration law Monday – a modest policy victory for the Obama Administration. But on the more important matter of the Constitution, the decision was an 8-0 defeat for the Administration's effort to upset the balance of power between the federal government and the states.


In Arizona v. United States, the majority overturned three of the four contested provisions of Arizona's controversial plan to have state and local police enforce federal immigration law. The Constitutional principles that Washington alone has the power to ``establish a uniform Rule of Naturalization'' and that federal laws precede state laws are noncontroversial. Arizona had attempted to fashion state policies that ran parallel to the existing federal ones.


Justice Anthony Kennedy, joined by Chief Justice John Roberts and the Court's liberals, ruled that the state flew too close to the federal sun. On the overturned provisions the majority held Congress had deliberately ``occupied the field'' and Arizona had thus intruded on the federal's privileged powers.


However, the Justices said that Arizona police would be allowed to verify the legal status of people who come in contact with law enforcement. That's because Congress has always envisioned joint federal-state immigration enforcement and explicitly encourages state officers to share information and cooperate with federal colleagues.


Two of the three objecting Justices – Samuel Alito and Clarence Thomas – agreed with this Constitutional logic but disagreed about which Arizona rules conflicted with the federal statute. The only major objection came from Justice Antonin Scalia, who offered an even more robust defense of state privileges going back to the Alien and Sedition Acts.


The 8-0 objection to President Obama turns on what Justice Samuel Alito describes in his objection as ``a shocking assertion of federal executive power''. The White House argued that Arizona's laws conflicted with its enforcement priorities, even if state laws complied with federal statutes to the letter. In effect, the White House claimed that it could invalidate any otherwise legitimate state law that it disagrees with.


Some powers do belong exclusively to the federal government, and control of citizenship and the borders is among them. But if Congress wanted to prevent states from using their own resources to check immigration status, it could. It never did so. The Administration was in essence asserting that because it didn't want to carry out Congress's immigration wishes, no state should be allowed to do so either. Every Justice rightly rejected this remarkable claim.