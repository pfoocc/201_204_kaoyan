\item Three provisions of Arizona's plan were overturned because they
\begin{tasks}
	\task disturbed the power balance between different states.
	\task overstepped the authority of federal immigration law.
	\task deprived the federal police of Constitutional powers.
	\task contradicted both the federal and state policies.
\end{tasks}
\item On which of the following did the Justices agree, according to Paragraph 4?
\begin{tasks}
	\task Congress's intervention in immigration enforcement.
	\task Federal officers' duty to withhold immigrants' information.
	\task States' legitimate role in immigration enforcement.
	\task States' independence from federal immigration law.
\end{tasks}
\item It can be inferred from Paragraph 5 that the Alien and Sedition Acts
\begin{tasks}
	\task stood in favor of the states.
	\task supported the federal statute.
	\task undermined the states' interests.
	\task violated the Constitution.
\end{tasks}
\item The White House claims that its power of enforcement
\begin{tasks}
	\task is dependent on the states' support.
	\task is established by federal statutes.
	\task outweighs that held by the states.
	\task rarely goes against state laws.
\end{tasks}
\item What can be learned from the last paragraph?
\begin{tasks}
	\task Immigration issues are usually decided by Congress.
	\task The Administration is dominant over immigration issues.
	\task Justices wanted to strengthen its coordination with Congress.
	\task Justices intended to check the power of the Administration.
\end{tasks}
