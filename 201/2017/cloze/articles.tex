Could a hug a day keep the doctor away? The answer may be a resounding ``yes!'' \uline{~~1~~} helping you feel close and \uline{~~2~~} to people you care about, it   turns out that hugs can bring a \uline{~~3~~} of health benefits to your body and mind.  Believe it or not, a warm embrace might even help you \uline{~~4~~} getting sick this winter.


In a recent study \uline{~~5~~} over 400 healthy adults, researchers from Carnegie  Mellon University in Pennsylvania examined the effects of perceived social support and the receipt of hugs \uline{~~6~~} the participants' susceptibility to  developing the common cold after being \uline{~~7~~} to the virus. People who  perceived greater social support were less likely to come \uline{~~8~~} with a cold, and  the researchers \uline{~~9~~} that the stress-reducing effects of hugging \uline{~~10~~} about  32 percent of that beneficial effect. \uline{~~11~~} among those who got a cold, the ones who felt greater social support and received more frequent hugs had less severe \uline{~~12~~}.


``Hugging protects people who are under stress from the \uline{~~13~~} risk for colds that's usually \uline{~~14~~} with stress,'' notes Sheldon Cohen, a professor of psychology at Carnegie. Hugging ``is a marker of intimacy and helps \uline{~~15~~} the  feeling that others are there to help \uline{~~16~~} difficulty.''


Some experts \uline{~~17~~} the stress-reducing, health-related benefits of hugging to the release of oxytocin, often  called ``the bonding hormone'' \uline{~~18~~} it promotes attachment in relationships, including that between mothers and their newborn babies. Oxytocin is made primarily in the central lower part of the brain, and some of it is released into the bloodstream. But some of it \uline{~~19~~} in the  brain, where it \uline{~~20~~} mood, behavior and physiology.