\item Queen Liliuokalani's remark in Paragraph 1 indicates
\begin{tasks}
	\task her conservative view on the historical role of astronomy.
	\task the importance of astronomy in ancient Hawaiian society.
	\task the regrettable decline of astronomy in ancient times.
	\task her appreciation of star watchers' feats in her time.
\end{tasks}
\item Mauna Kea is deemed as an ideal astronomical site due to
\begin{tasks}
	\task its geographical features.
	\task its protective surroundings.
	\task its religious implications.
	\task its existing infrastructure.
\end{tasks}
\item The construction of the TMT is opposed by some locals partly because
\begin{tasks}
	\task it may risk ruining their intellectual life.
	\task it reminds them of a humiliating history.
	\task their culture will lose a chance of revival.
	\task they fear losing control of Mauna Kea.
\end{tasks}
\item It can be inferred from Paragraph 5 that progress in today's astronomy
\begin{tasks}
	\task is fulfilling the dreams of ancient Hawaiians.
	\task helps spread Hawaiian culture across the world.
	\task may uncover the origin of Hawaiian culture.
	\task will eventually soften Hawaiians' hostility.
\end{tasks}
\item The author's attitude toward choosing Mauna Kea as the TMT site is one of
\begin{tasks}
	\task severe criticism.
	\task passive acceptance.
	\task slight hesitancy.
	\task full approval.
\end{tasks}