\item The underlined sentence (Para.1) most probably shows that the court
\begin{tasks}
	\task avoided defining the extent of McDonnell's duties.
	\task made no compromise in convicting McDonnell.
	\task was contemptuous of McDonnell's conduct.
	\task refused to comment on McDonnell's ethics.
\end{tasks}
\item According to Paragraph 4, an official act is deemed corruptive only if it involves
\begin{tasks}
	\task leaking secrets intentionally.
	\task sizable gains in the form of gifts.
	\task concrete returns for gift-givers.
	\task breaking contracts officially.
\end{tasks}
\item The court's ruling is based on the assumption that public officials are
\begin{tasks}
	\task justified in addressing the needs of their constituents.
	\task qualified to deal independently with bureaucratic issues.
	\task allowed to focus on the concerns of their supporters.
	\task exempt from conviction on the charge of favoritism.
\end{tasks}
\item Well-enforced laws in government transparency are needed to
\begin{tasks}
	\task awaken the conscience of officials.
	\task guarantee fair play in official access.
	\task allow for certain kinds of lobbying.
	\task inspire hopes in average people.
\end{tasks}
\item The author's attitude toward the court's ruling is
\begin{tasks}
	\task sarcastic.
	\task tolerant.
	\task skeptical.
	\task supportive.
\end{tasks}