\item Robert F. Kennedy is cited because he
\begin{tasks}
	\task praised the UK for its GDP.
	\task identified GDP with happiness.
	\task misinterpreted the role of GDP.
	\task had a low opinion of GDP.
\end{tasks}
\item It can be inferred from Paragraph 2 that
\begin{tasks}
	\task the UK is reluctant to remold its economic pattern.
	\task GDP as the measure of success is widely defied in the UK.
	\task the UK will contribute less to the world economy.
	\task policymakers in the UK are paying less attention to GDP.
\end{tasks}
\item Which of the following is true about the recent annual study?
\begin{tasks}
	\task It is sponsored by 163 countries.
	\task It excludes GDP as an indicator.
	\task Its criteria are questionable.
	\task Its results are enlightening.
\end{tasks}
\item In the last two paragraphs, the author suggests that
\begin{tasks}
	\task the UK is preparing for an economic boom.
	\task high GDP foreshadows an economic decline.
	\task it is essential to consider factors beyond GDP.
	\task it requires caution to handle economic issues.
\end{tasks}
\item Which of the following is the best title for the text?
\begin{tasks}
	\task High GDP But Inadequate Well-being, a UK Lesson
	\task GDP Figures, a Window on Global Economic Health
	\task Robert F. Kennedy, a Terminator of GDP
	\task Brexit, the UK's Gateway to Well-being
\end{tasks}