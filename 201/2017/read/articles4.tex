In a rare unanimous ruling, the US Supreme Court has overturned the corruption conviction of a former Virginia governor, Robert McDonnell. But it did so while holding its nose at the ethics of his conduct, which included accepting gifts such as a Rolex watch and a Ferrari automobile from a company seeking access to government.


The high court's decision said the judge in Mr. McDonnell's trial failed to tell a jury that it must look only at his ``official acts,'' or the former governor's decisions on ``specific'' and ``unsettled'' issues related to his duties.


Merely helping a gift-giver gain access to other officials, unless done with clear intent to pressure those officials, is not corruption, the justices found.


The court did suggest that accepting favors in return for opening doors is ``distasteful'' and ``nasty.'' But under anti-bribery laws, proof must be made of concrete benefits, such as approval of a contract or regulation. Simply arranging a meeting, making a phone call, or hosting an event is not an ``official act''.


The court's ruling is legally sound in defining a kind of favoritism that is not criminal. Elected leaders must be allowed to help supporters deal with bureaucratic problems without fear of prosecution for bribery. ``The basic compact underlying representative government,'' wrote Chief Justice John Roberts for the court, ``assumes that public officials will hear from their constituents and act on their concerns.''


But the ruling reinforces the need for citizens and their elected representatives, not the courts, to ensure equality of access to government. Officials must not be allowed to play favorites in providing information or in arranging meetings simply because an individual or group provides a campaign donation or a personal gift. This type of integrity requires well-enforced laws in government transparency, such as records of official meetings, rules on lobbying, and information about each elected leader's source of wealth.


Favoritism in official access can fan public perceptions of corruption. But it is not always corruption. Rather officials must avoid double standards, or different types of access for average people and the wealthy. If connections can be bought, a basic premise of democratic society – that all are equal in treatment by government – is undermined. Good governance rests on an understanding of the inherent worth of each individual.


The court's ruling is a step forward in the struggle against both corruption and official favoritism.