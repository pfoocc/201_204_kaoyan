Robert F. Kennedy once said that a country's GDP measures ``everything except that which makes life worthwhile.'' With Britain voting to leave the European Union, and GDP already predicted to slow as a result, it is now a timely moment to assess what he was referring to.


The question of GDP and its usefulness has annoyed policymakers for over half a century. Many argue that it is a flawed concept. It measures things that do not matter and misses things that do. By most recent measures, the UK's GDP has been the envy of the Western world, with record low unemployment and high growth figures. If everything was going so well, then why did over 17 million people vote for Brexit, despite the warnings about what it could do to their country's economic prospects?


A recent annual study of countries and their ability to convert growth into well-being sheds some light on that question. Across the 163 countries measured, the UK is one of the poorest performers in ensuring that economic growth is translated into meaningful improvements for its citizens. Rather than just focusing on GDP, over 40 different sets of criteria from health, education and civil society engagement have been measured to get a more rounded assessment of how countries are performing.


While all of these countries face their own challenges, there are a number of consistent themes. Yes, there has been a budding economic recovery since the 2008 global crash, but in key indicators in areas such as health and education, major economies have continued to decline. Yet this isn't the case with all countries. Some relatively poor European countries have seen huge improvements across measures including civil society, income equality and the environment.


This is a lesson that rich countries can learn: When GDP is no longer regarded as the sole measure of a country's success, the world looks very different.


So, what Kennedy was referring to was that while GDP has been the most common method for measuring the economic activity of nations, as a measure, it   is no longer enough. It does not include important factors such as environmental quality or education outcomes – all things that contribute to a person's sense of well-being.


The sharp hit to growth predicted around the world and in the UK could lead   to a decline in the everyday services we depend on for our well-being and for growth. But policymakers who refocus efforts on improving well-being rather than simply worrying about GDP figures could avoid the forecasted doom and may even see progress.