\item According to the first paragraph, peer pressure often emerges as
\begin{tasks}
	\task a supplement to the social cure.
	\task a stimulus to group dynamics.
	\task an obstacle to social progress.
	\task a cause of undesirable behaviors.
\end{tasks}
\item Rosenberg holds that public-health advocates should
\begin{tasks}
	\task recruit professional advertisers.
	\task learn from advertisers' experience.
	\task stay away from commercial advertisers.
	\task recognize the limitations of advertisements.
\end{tasks}
\item In the author's view, Rosenberg's book fails to
\begin{tasks}
	\task adequately probe social and biological factors.
	\task effectively evade the flaws of the social cure.
	\task illustrate the functions of state funding.
	\task produce a long-lasting social effect.
\end{tasks}
\item Paragraph 5 shows that our imitation of behaviors
\begin{tasks}
	\task is harmful to our networks of friends.
	\task will mislead behavioral studies.
	\task occurs without our realizing it.
	\task can produce negative health habits.
\end{tasks}
\item The author suggests in the last paragraph that the effect of peer pressure is
\begin{tasks}
	\task harmful.
	\task desirable.
	\task profound.
	\task questionable.
\end{tasks}