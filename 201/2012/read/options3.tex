\item According to the first paragraph, the process of discovery is characterized by its
\begin{tasks}
	\task uncertainty and complexity.
	\task misconception and deceptiveness.
	\task logicality and objectivity.
	\task systematicness and regularity.
\end{tasks}
\item It can be inferred from Paragraph 2 that credibility process requires
\begin{tasks}
	\task strict inspection.
	\task shared efforts.
	\task individual wisdom.
	\task persistent innovation.
\end{tasks}
\item Paragraph 3 shows that a discovery claim becomes credible after it
\begin{tasks}
	\task has attracted the attention of the general public.
	\task has been examined by the scientific community.
	\task has received recognition from editors and reviewers.
	\task has been frequently quoted by peer scientists.
\end{tasks}
\item Albert Szent-Györgyi would most likely agree that
\begin{tasks}
	\task scientific claims will survive challenges.
	\task discoveries today inspire future research.
	\task efforts to make discoveries are justified.
	\task scientific work calls for a critical mind.
\end{tasks}
\item Which of the following would be the best title of the text?
\begin{tasks}
	\task Novelty as an Engine of Scientific Development.
	\task Collective Scrutiny in Scientific Discovery.
	\task Evolution of Credibility in Doing Science.
	\task Challenge to Credibility at the Gate to Science.
\end{tasks}