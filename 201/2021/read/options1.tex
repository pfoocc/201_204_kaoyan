\item The author holds that this year's increase in rail passenger fares
\begin{tasks}
	\task has kept pace with inflation.
	\task is a big surprise to commuters.
	\task remains an unreasonable measure.
	\task will ease train operators' burden.
\end{tasks}
\item The stockbroker in Paragraph 2 is used to stand for
\begin{tasks}
	\task car drivers.
	\task rail travelers.
	\task local investors.
	\task ordinary taxpayers.
\end{tasks}
\item It is indicated in Paragraph 3 that train operators
\begin{tasks}
	\task are offering compensation to commuters.
	\task are trying to repair relations with the unions.
	\task have failed to provide an adequate service.
	\task have suffered huge losses owing to the strikes.
\end{tasks}
\item If unable to calm down passengers, the railways may have to face
\begin{tasks}
	\task the loss of investment.
	\task the collapse of operations.
	\task a reduction of revenue.
	\task a change of ownership.
\end{tasks}
\item Which of the following would be the best title for the text?
\begin{tasks}
	\task Who Are to Blame for the Strikes?
	\task Constant Complaining Doesn't Work
	\task Can Nationalisation Bring Hope?
	\task Ever-rising Fares Aren't Sustainable
\end{tasks}