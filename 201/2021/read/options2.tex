\item According to the first two paragraphs, CCT programs aim to
\begin{tasks}
	\task facilitate healthcare reform.
	\task help poor families get better off.
	\task improve local education systems.
	\task lower deforestation rates.
\end{tasks}
\item The study based on an area in Mexico is cited to show that
\begin{tasks}
	\task cattle raising has been a major means of livelihood for the poor.
	\task CCT programs have helped preserve traditional lifestyles.
	\task antipoverty efforts require the participation of local farmers.
	\task economic growth tends to cause environmental degradation.
\end{tasks}
\item In his study about Indonesia, Ferraro intends to find out
\begin{tasks}
	\task its acceptance level of CCTs.
	\task its annual rate of poverty alleviation.
	\task the relation of CCTs to its forest loss.
	\task the role of its forests in climate change.
\end{tasks}
\item According to Ferraro, the CCT program in Indonesia is most valuable in that
\begin{tasks}
	\task it will benefit other Asian countries.
	\task it will reduce regional inequality.
		[C]it can protect the environment.
	\task it can benefit grain production.
\end{tasks}
\item What is the text centered on?
\begin{tasks}
	\task The effects of a program.
	\task The debates over a program.
	\task The process of a study.
	\task The transferability of a study.
\end{tasks}