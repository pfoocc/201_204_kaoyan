\item According to Paragraph 1, the author's posts on Twitter
\begin{tasks}
	\task changed people's impression of the Victorians.
	\task highlighted social media's role in Victorian studies.
	\task re-evaluated the Victorian's notion of public image.
	\task illustrated the development of Victorian photography.
\end{tasks}
\item What does the author say about the Victorian portraits he has collected?
\begin{tasks}
	\task They are in popular use among historians.
	\task They are rare among photographs of that age.
	\task They mirror 19th-century social conventions.
	\task They show effects of different exposure times.
\end{tasks}
\item What might have kept the Victorians from smiling for pictures in the 1890s?
\begin{tasks}
	\task Their inherent social sensitiveness.
	\task Their tension before the camera.
	\task Their distrust of new inventions.
	\task Their unhealthy dental condition.
\end{tasks}
\item Mark Twain is quoted to show that the disapproval of smiles in pictures was
\begin{tasks}
	\task a deep-root belief.
	\task a misguided attitude.
	\task a controversial view.
	\task a thought-provoking idea.
\end{tasks}
\item Which of the following questions does the text answer?
\begin{tasks}
	\task Why did most Victorians look stern in photographs?
	\task Why did the Victorians start to view photographs?
	\task What made photography develop in the Victorian period?
	\task How did smiling in photographs become a post-Victorian norm?
\end{tasks}