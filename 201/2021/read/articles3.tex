As a historian who's always searching for the text or the image that makes us re-evaluate the past, I've become preoccupied with looking for photographs that show our Victorian ancestors smiling ( what better way to shatter the image of 19th-century prudery?). I've found quite a few, and—since I started posting them on Twitter—they have been causing quite a stir. People have been surprised to see evidence that Victorians had fun and could, and did, laugh. They are noting that the Victorians suddenly seem to become more human as the hundred-or-so years that separate us fade away through our common experience of laughter.


Of course, I need to concede that my collection of `Smiling Victorians' makes up only a tiny percentage of the vast catalogue of photographic portraiture created between 1840 and 1900, the majority of which show sitters posing miserably and stiffly in front of painted backdrops, or staring absently into the middle distance. How do we explain this trend?


During the 1840s and 1850s, in the early days of photography, exposure times were notoriously long: the daguerreotype photographic method (producing an image on a silvered copper plate) could take several minutes to complete, resulting in blurred images as sitters shifted position or adjusted their limbs. The thought of holding a fixed grin as the camera performed its magical duties was too much to contemplate, and so anon-committal blank stare became the norm.


But exposure times were much quicker by the 1880s, and the introduction of the Box Brownie and other portable cameras meant that, though slow by today's digital standards, the exposure was almost instantaneous. Spontaneous smiles were relatively easy to capture by the 1890s, so we must look elsewhere for an explanation of why Victorians still hesitated to smile.


One explanation might be the loss of dignity displayed through a cheesy grin. ``Nature gave us lips to conceal our teeth,'' ran one popular Victorian maxim, alluding to the fact that before the birth of proper dentistry, mouths were often in a shocking state of hygiene. A flashing set of healthy and clean, regular `pearly whites' was a rare sight in Victorian society, the preserve of the super-rich (and even then, dental hygiene was not guaranteed).


A toothy grin ( especially when there were gaps or blackened gnashers) lacked class: drunks, tramps, and music hall performers might gurn and grin with a smile as wide as Lewis Carroll's gum-exposing Cheshire Cat, but it was not a becoming look for properly bred persons. Even Mark Twain, a man who enjoyed a hearty laugh, said that when it came to photographic portraits there could be ``nothing more damning than a silly, foolish smile fixed forever''.