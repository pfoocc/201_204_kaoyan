Fluid intelligence is the type of intelligence that has to do with short-term memory and the ability to think quickly, logically, and abstractly in order to solve new problems. It \uline{~~1~~} in young adulthood, levels out for a period of time, and then \uline{~~2~~} starts to slowly decline as we age. But \uline{~~3~~} aging is inevitable, scientists are finding out that certain changes in brain function may not be.


One study found that muscle loss and the \uline{~~4~~} of body fat around the abdomen are associated with a decline in fluid intelligence. This suggests the \uline{~~5~~} that lifestyle factors might help prevent or \uline{~~6~~} this type of decline.


The researchers looked at data that \uline{~~7~~} measurements of lean muscle and abdominal fat from more than 4,000 middle-to-older-aged men and women and \uline{~~8~~} that data to reported changes in fluid intelligence over a six-year period. They found that middle-aged people \uline{~~9~~} higher measures of abdominal fat \uline{~~10~~} worse on measures of fluid intelligence as the years \uline{~~11~~}.


For women, the association may be \uline{~~12~~} to changes in immunity that resulted from excess abdominal fat; in men, the immune system did not appear to be \uline{~~13~~}. It is hoped that future studies could \uline{~~14~~} these differences and perhaps lead to different \uline{~~15~~} for men and women.


\uline{~~16~~} there are steps you can \uline{~~17~~} to help reduce abdominal fat and maintain lean muscle mass as you age in order to protect both your physical and mental \uline{~~18~~} The two highly recommended lifestyle approaches are maintaining or increasing your \uline{~~19~~} of aerobic exercise and following Mediterranean-style \uline{~~20~~} that is high in fiber and eliminates highly processed foods.