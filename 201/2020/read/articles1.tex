A group of Labour MPs, among them Yvette Cooper, are bringing in the new year with a call to institute a UK ``town of culture'' award. The proposal is that it should sit alongside the existing city of culture title, which was held by Hull in 2017, and has been awarded to Coventry for 2021. Cooper and her colleagues argue that the success of the crown for Hull, where it brought in £220m of investment and an avalanche of arts, ought not to be confined to cities. Britain's towns, it is true, are not prevented from applying, but they generally lack the resources to put together a bid to beat their bigger competitors. A town of culture award could, it is argued, become an annual event, attracting funding and creating jobs.


Some might see the proposal as a booby prize for the fact that Britain is no longer able to apply for the much more prestigious title of European capital of culture, a sought-after award bagged by Glasgow in 1990 and Liverpool in 2008. A cynic might speculate that the UK is on the verge of disappearing into an endless fever of self-celebration in its desperation to reinvent itself for the post-Brexit world: after town of culture, who knows what will follow-village of culture? Suburb of culture? Hamlet of culture?


It is also wise to recall that such titles are not a cure-all. A badly run ``year of culture'' washes in and washes out of a place like the tide, bringing prominence for a spell but leaving no lasting benefits to the community. The really successful holders of such titles are those that do a great deal more than fill hotel bedrooms and bring in high-profile arts events and good press for a year. They transform the aspirations of the people who live there; they nudge the self-image of the city into a bolder and more optimistic light. It is hard to get right, and requires a remarkable degree of vision, as well as cooperation between city authorities, the private sector, community groups and cultural organisations. But it can be done: Glasgow's year as European capital of culture can certainly be seen as one of a complex series of factors that have turned the city into the powerhouse of art, music and theatre that it remains today.


A ``town of culture'' could be not just about the arts but about honouring a town's peculiarities—helping sustain its high street, supporting local facilities and above all celebrating its people. Jeremy Wright, the culture secretary, should welcome this positive, hope-filled proposal, and tum it into action.