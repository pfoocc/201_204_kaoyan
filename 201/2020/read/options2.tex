\item Scientific publishing is seen as ``a licence to print money'' partly because
\begin{tasks}
	\task its funding has enjoyed a steady increase.
	\task its marketing strategy has been successful.
	\task its payment for peer review is reduced.
	\task its content acquisition costs nothing.
\end{tasks}
\item According to Paragraphs 2 and 3, scientific publishers Elsevier have
\begin{tasks}
	\task thrived mainly on university libraries.
	\task gone through an existential crisis.
	\task revived the publishing industry.
	\task financed researchers generously.
\end{tasks}
\item How does the author feel about the success of Sci-Hub?
\begin{tasks}
	\task Relieved.
	\task Puzzled.
	\task Concerned.
	\task Encouraged.
\end{tasks}
\item It can be learned from Paragraphs 5 and 6 that open access terms
\begin{tasks}
	\task allow publishers some room to make money.
	\task render publishing much easier for scientists.
	\task reduce the cost of publication substantially.
	\task free universities from financial burdens.
\end{tasks}
\item Which of the following characterizes the scientific publishing model?
\begin{tasks}
	\task Trial subscription is offered.
	\task Labour triumphs over status.
	\task Costs are well controlled.
	\task The few feed on the many.
\end{tasks}