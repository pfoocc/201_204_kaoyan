Even if families don't sit down to eat together as frequently as before, millions of Britons will nonetheless have got a share this weekend of one of that nation's great traditions:the Sunday roast. \uline{~~1~~} a cold winter's day, few culinary pleasures can \uline{~~2~~} it. Yet as we report now, the food police are determined that this \uline{~~3~~} should be rendered yet another guilty pleasure \uline{~~4~~} to damage our health.


The Food Standards Authority (FSA) has \uline{~~5~~} a public warning about the risks of a compound called acrylamide that forms in some foods cooked \uline{~~6~~} high temperatures. This means that people should \uline{~~7~~} crisping their roast potatoes, reject thin- crust pizzas and only \uline{~~8~~} toast their bread. But where is the evidence to support such alarmist advice? \uline{~~9~~} studies have shown that acrylamide can cause neurological damage in mice, there is no \uline{~~10~~} evidence that it causes cancer in humans.


Scientists say the compound is \uline{~~11~~} to cause cancer but have no hard scientific proof \uline{~~12~~} the precautionary principle, it could be argued that it is \uline{~~13~~} to follow the FSA advice. \uline{~~14~~}, it was rumoured that smoking caused cancer for years before the evidence was found to prove a \uline{~~15~~}.


Doubtless a piece of boiled beef can always be \uline{~~16~~} up on Sunday alongside some steamed vegetables, without the Yorkshire pudding and no wine. But would life be worth living? \uline{~~17~~}, the FSA says it is not telling people to cut out roast foods \uline{~~18~~}, but to reduce their lifetime intake. However, their \uline{~~19~~} risks coming a cross as being pushy and overprotective.Constant health scares just \uline{~~20~~} with  one listening.