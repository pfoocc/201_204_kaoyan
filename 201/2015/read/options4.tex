\item According to the first two paragraphs, Elisabeth was upset by
\begin{tasks}
	\task the consequences of the current sorting mechanism.
	\task companies' financial loss due to immoral practices.
	\task governmental ineffectiveness on moral issues.
	\task the wide misuse of integrity among institutions.
\end{tasks}
\item It can be inferred from Paragraph 3 that
\begin{tasks}
	\task Glenn Mulcaire may deny phone hacking as a crime.
	\task more journalists may be found guilty of phone hacking.
	\task Andy Coulson should be held innocent of the charge.
	\task phone hacking will be accepted on certain occasions.
\end{tasks}
\item The author believes that Rebekah Brooks's defence
\begin{tasks}
	\task was hardly convincing.
	\task centered on trivial issues.
	\task revealed a cunning personality.
	\task was part of a conspiracy.
\end{tasks}
\item The author holds that the current collective doctrine shows
\begin{tasks}
	\task a marginalized lifestyle.
	\task unfair wealth distribution.
	\task generally distorted values.
	\task a rigid moral code.
\end{tasks}
\item Which of the following is suggested in the last paragraph?
\begin{tasks}
	\task The quality of writings is of primary importance.
	\task Moral awareness matters in editing a newspaper.
	\task Common humanity is central to news reporting.
	\task Journalists need stricter industrial regulations.
\end{tasks}