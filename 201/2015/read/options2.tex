\item The Supreme Court will work out whether, during an arrest, it is legitimate to
\begin{tasks}
	\task prevent suspects from deleting their phone contents.
	\task search for suspects' mobile phones without a warrant.
	\task check suspects' phone contents without being authorized.
	\task prohibit suspects from using their mobile phones.
\end{tasks}
\item The author's attitude toward California's argument is one of
\begin{tasks}
	\task disapproval.
	\task indifference.
	\task tolerance.
	\task cautiousness.
\end{tasks}
\item The author believes that exploring one's phone contents is comparable to
\begin{tasks}
	\task going through one's wallet.
	\task handling one's historical records.
	\task scanning one's correspondences.
	\task getting into one's residence.
\end{tasks}
\item In Paragraphs 5 and 6, the author shows his concern that
\begin{tasks}
	\task principles are hard to be clearly expressed.
	\task the court is giving police less room for action.
	\task phones are used to store sensitive information.
	\task citizens' privacy is not effectively protected.
\end{tasks}
\item Orin Kerr's comparison is quoted to indicate that
\begin{tasks}
	\task the Constitution should be implemented flexibly.
	\task principles of the Constitution should never be altered.
	\task California's argument violates principles of the Constitution.
	\task new technology requires reinterpretation of the Constitution.
\end{tasks}