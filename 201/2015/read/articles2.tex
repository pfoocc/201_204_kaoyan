Just how much does the Constitution protect your digital data? The Supreme Court will now consider whether police can search the contents of a mobile phone without a warrant if the phone is on or around a person during an arrest.


California has asked the justices to refrain from a sweeping ruling, particularly one that upsets the old assumption that authorities may search through the possessions of suspects at the time of their arrest. It is hard, the state argues, for judges to assess the implications of new and rapidly changing technologies.


The court would be recklessly modest if it followed California's advice. Enough of the implications are discernable, even obvious, so that the justices can and should provide updated guidelines to police, lawyers and defendants.


They should start by discarding California's lame argument that exploring the contents of a smartphone – a vast storehouse of digital information – is similar to, say, going through a suspect's purse. The court has ruled that police don't violate the Fourth Amendment when they go through the wallet or pocketbook of an arrestee without a warrant. But exploring one's smartphone is more like entering  his or her home. A smartphone may contain an arrestee's reading history, financial history, medical history and comprehensive records of recent correspondence. The development of ``cloud computing'', meanwhile, has made that exploration so much the easier.


Americans should take steps to protect their digital privacy. But keeping sensitive information on these devices is increasingly a requirement of normal life. Citizens still have a right to expect private documents to remain private and protected by the Constitution's prohibition on unreasonable searches.


As so often is the case, stating that principle doesn't ease the challenge of line-drawing. In many cases, it would not be overly burdensome for authorities to obtain a warrant to search through phone contents. They could still invalidate Fourth Amendment protections when facing severe, urgent circumstances, and they could take reasonable measures to ensure that phone data are not erased or altered while waiting for a warrant. The court, though, may want to allow room for police to cite situations where they are entitled to more freedom.


But the justices should not swallow California's argument whole. New, disruptive technology sometimes demands novel applications of the Constitution's protections. Orin Kerr, a law professor, compares the explosion and accessibility of digital information in the 21st century with the establishment of automobile use as a virtual necessity of life in the 20th: The justices had to specify novel rules for the new personal domain of the passenger car then; they must sort out how the Fourth Amendment applies to digital information now.


