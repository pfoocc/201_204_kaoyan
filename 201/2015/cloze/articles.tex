Though not biologically related, friends are as ``related'' as fourth cousins, sharing about 1\% of genes. That is \uline{~~1~~} a study, published from the University  of California and Yale University in the Proceedings of the National Academy of Sciences, has \uline{~~2~~}.


The study is a genome-wide analysis conducted \uline{~~3~~} 1,932 unique subjects which \uline{~~4~~} pairs of unrelated friends and unrelated strangers. The same people were used in both \uline{~~5~~}.


While 1\% may seem \uline{~~6~~}, it is not so to a geneticist. As James Fowler, professor of medical genetics at UC San Diego, says, ``Most people do not even \uline{~~7~~} their fourth cousins but somehow manage to select as friends the people who \uline{~~8~~} our kin.''


The study \uline{~~9~~} found that the genes for smell were something shared in friends but not genes for immunity. Why this similarity exists in smell genes is difficult to explain, for now. \uline{~~10~~}, as the team suggests, it draws us to similar environments but there is more \uline{~~11~~} it. There could be many mechanisms  working together that \uline{~~12~~} us in choosing genetically similar friends \uline{~~13~~} ``functional kinship'' of being friends with \uline{~~14~~} !


One of the remarkable findings of the study was that the similar genes seem  to be evolving \uline{~~15~~} than other genes. Studying this could help \uline{~~16~~} why  human evolution picked pace in the last 30,000 years, with social environment being a major \uline{~~17~~} factor.


The findings do not simply explain people's \uline{~~18~~} to befriend those of  similar \uline{~~19~~} backgrounds, say the researchers. Though all the subjects were  drawn from a population of European extraction, care was taken to \uline{~~20~~} that all subjects, friends and strangers, were taken from the same population.