Trust is a tricky business. On the one hand, it's a necessary condition \uline{~~1~~} many worthwhile things: child care, friendships, etc. On the other hand, putting your \uline{~~2~~} in the wrong place often carries a high \uline{~~3~~}.


\uline{~~4~~}, why do we trust at all? Well, because it feels good. \uline{~~5~~} people place their trust in an individual or an institution, their brains release oxytocin, a hormone that \uline{~~6~~} pleasurable feelings and triggers the herding instinct that prompts  humans to \uline{~~7~~} with one another. Scientists have found that exposure \uline{~~8~~} this  hormone puts us in a trusting \uline{~~9~~} : In a Swiss study, researchers sprayed oxytocin into the noses of half the subjects; those subjects were ready to lend significantly higher amounts of money to strangers than were their \uline{~~10~~} who inhaled something else.


\uline{~~11~~} for us, we also have a sixth sense for dishonesty that may \uline{~~12~~} us. A Canadian study found that children as young as 14 months can differentiate \uline{~~13~~} a credible person and a dishonest one. Sixty toddlers were each \uline{~~14~~} to an adult  tester holding a plastic container. The tester would ask, ``What's in here?'' before looking into the container, smiling, and exclaiming, ``Wow!'' Each subject was then invited to look \uline{~~15~~}. Half of them found a toy; the other half \uline{~~16~~} the  container was empty – and realized the tester had \uline{~~17~~} them.


Among the children who had not been tricked, the majority were \uline{~~18~~} to cooperate with the tester in learning a new skill, demonstrating that they trusted his leadership. \uline{~~19~~}, only five of the 30 children paired with the `` \uline{~~20~~} '' tester participated in a follow-up activity.