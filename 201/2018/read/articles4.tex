The U.S. Postal Service (USPS) continues to bleed red ink. It reported a net loss of \$5.6 billion for fiscal 2016, the 10th straight year its expenses have exceeded revenue. Meanwhile, it has more than \$120 billion in unfunded liabilities, mostly for employee health and retirement costs. There are many reasons this formerly stable federal institution finds itself at the brink of bankruptcy. Fundamentally, the USPS is in a historic squeeze between technological change that has permanently decreased demand for its bread-and-butter product, first-class mail, and a regulatory structure that denies management the flexibility to adjust its operations to the new reality.


And interest groups ranging from postal unions to greeting-card makers exert self-interested pressure on the USPS's ultimate overseer – Congress – insisting that whatever else happens to the Postal Service, aspects of the status quo they depend on get protected. This is why repeated attempts at reform legislation have failed in recent years, leaving the Postal Service unable to pay its bills except by deferring vital modernization.


Now comes word that everyone involved – Democrats, Republicans, the Postal Service, the unions and the system's heaviest users – has finally agreed on a plan to fix the system. Legislation is moving through the House that would save USPS an estimated \$28.6 billion over five years, which could help pay for new vehicles, among other survival measures. Most of the money would come from a penny-per-letter permanent rate increase and from shifting postal retirees into Medicare. The latter step would largely offset the financial burden of annually pre-funding retiree health care, thus addressing a long-standing complaint by the USPS and its unions.


If it clears the House, this measure would still have to get through the Senate – where someone is bound to point out that it amounts to the bare, bare minimum necessary to keep the Postal Service afloat, not comprehensive reform. There's no change to collective bargaining at the USPS, a major omission considering that personnel accounts for 80 percent of the agency's costs. Also missing is any discussion of eliminating Saturday letter delivery. That common-sense change enjoys wide public support and would save the USPS \$2 billion per year. But postal special-interest groups seem to have killed it, at least in the House. The emerging consensus around the bill is a sign that legislators are getting frightened about a politically embarrassing short-term collapse at the USPS. It is not, however, a sign that they're getting serious about transforming the postal system for the 21st century.


