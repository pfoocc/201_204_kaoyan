Among the annoying challenges facing the middle class is one that will probably go unmentioned in the next presidential campaign: What happens when the robots come for their jobs?


Don't dismiss that possibility entirely. About half of U.S. jobs are at high risk of being automated, according to a University of Oxford study, with the middle class disproportionately squeezed. Lower-income jobs like gardening or day care don't appeal to robots. But many middle-class occupations – trucking, financial advice, software engineering – have aroused their interest, or soon will. The rich own the robots, so they will be fine.


This isn't to be alarmist. Optimists point out that technological upheaval has benefited workers in the past. The Industrial Revolution didn't go so well for Luddites whose jobs were displaced by mechanized looms, but it eventually raised living standards and created more jobs than it destroyed. Likewise, automation should eventually boost productivity, stimulate demand by driving down prices, and free workers from hard, boring work. But in the medium term, middle-class workers may need a lot of help adjusting.


The first step, as Erik Brynjolfsson and Andrew McAfee argue in The Second Machine Age, should be rethinking education and job training. Curriculums – from grammar school to college – should evolve to focus less on memorizing facts and more on creativity and complex communication. Vocational schools should do a better job of fostering problem-solving skills and helping students work alongside robots. Online education can supplement the traditional kind. It could make extra training and instruction affordable. Professionals trying to acquire new skills will be able to do so without going into debt.


The challenge of coping with automation underlines the need for the U.S. to revive its fading business dynamism: Starting new companies must be made easier. In previous eras of drastic technological change, entrepreneurs smoothed the transition by dreaming up ways to combine labor and machines. The best uses of 3D printers and virtual reality haven't been invented yet. The U.S. needs the new companies that will invent them.


Finally, because automation threatens to widen the gap between capital income and labor income, taxes and the safety net will have to be rethought. Taxes on low-wage labor need to be cut, and wage subsidies such as the earned income tax credit should be expanded: This would boost incomes, encourage work, reward companies for job creation, and reduce inequality.


Technology will improve society in ways big and small over the next few years, yet this will be little comfort to those who find their lives and careers upended by automation. Destroying the machines that are coming for our jobs would be nuts. But policies to help workers adapt will be indispensable.