\item According to Paragraphs 1 and 2, many young Americans cast doubts on
\begin{tasks}
	\task the justification of the news-filtering practice.
	\task people's preference for social media platforms.
	\task the administration's ability to handle information.
	\task social media as a reliable source of news.
\end{tasks}
\item The phrase ``beef up'' (Line 2, Para. 2) is closest in meaning to
\begin{tasks}
	\task sharpen.
	\task define.
	\task boast.
	\task share.
\end{tasks}
\item According to the Knight Foundation survey, young people
\begin{tasks}
	\task tend to voice their opinions in cyberspace.
	\task verify news by referring to diverse sources.
	\task have a strong sense of responsibility.
	\task like to exchange views on ``distributed trust''.
\end{tasks}
\item The Barna survey found that a main cause for the fake news problem is
\begin{tasks}
	\task readers' outdated values.
	\task journalists' biased reporting.
	\task readers' misinterpretation.
	\task journalists' made-up stories.
\end{tasks}
\item Which of the following would be the best title for the text?
\begin{tasks}
	\task A Rise in Critical Skills for Sharing News Online.
	\task A Counteraction Against the Over-tweeting Trend.
	\task The Accumulation of Mutual Trust on Social Media.
	\task The Platforms for Projection of Personal Interests.
\end{tasks}
