\item Who will be most threatened by automation?
\begin{tasks}
	\task Leading politicians.
	\task Low-wage laborers.
	\task Robot owners.
	\task Middle-class workers.
\end{tasks}
\item Which of the following best represents the author's view?
\begin{tasks}
	\task Worries about automation are in fact groundless.
	\task Optimists' opinions on new tech find little support.
	\task Issues arising from automation need to be tackled.
	\task Negative consequences of new tech can be avoided.
\end{tasks}
\item Education in the age of automation should put more emphasis on
\begin{tasks}
	\task creative potential.
	\task job-hunting skills.
	\task individual needs.
	\task cooperative spirit.
\end{tasks}
\item The author suggests that tax policies be aimed at
\begin{tasks}
	\task encouraging the development of automation.
	\task increasing the return on capital investment.
	\task easing the hostility between rich and poor.
	\task preventing the income gap from widening.
\end{tasks}
\item In this text, the author presents a problem with
\begin{tasks}
	\task opposing views on it.
	\task possible solutions to it.
	\task its alarming impacts.
	\task its major variations.
\end{tasks}
