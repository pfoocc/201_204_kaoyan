By 1830 the former Spanish and Portuguese colonies had become independent nations. The roughly 20 million \uline{~~1~~} of these nations looked \uline{~~2~~} to the future. Born in the crisis of the old regime and Iberian colonialism, many of the leaders of independence \uline{~~3~~} the ideals of representative government, careers \uline{~~4~~} to  talent, freedom of commerce and trade, the \uline{~~5~~} to private property, and a belief in the individual as the basis of society. \uline{~~6~~} there was a belief that the new nations should be sovereign and independent states, large enough to be economically viable and integrated by a \uline{~~7~~} set of laws.


On the issue of \uline{~~8~~} of religion and the position of the Church, \uline{~~9~~},   there was less agreement \uline{~~10~~} the leadership. Roman Catholicism had been  the state religion and the only one \uline{~~11~~} by the Spanish crown. \uline{~~12~~} most leaders sought to maintain Catholicism \uline{~~13~~} the official religion of the new states, some sought to end the \uline{~~14~~} of other faiths. The defense of the Church  became a rallying \uline{~~15~~} for the conservative forces.


The ideals of the early leaders of independence were often egalitarian, valuing equality of everything. Bolivar had received aid from Haiti and had \uline{~~16~~} in return to abolish slavery in the areas he liberated. By 1854 slavery had been abolished everywhere except Spain' s \uline{~~17~~} colonies. Early promises to end Indian tribute  and taxes on people of mixed origin came much \uline{~~18~~} because the new nations still needed the revenue such policies \uline{~~19~~}. Egalitarian sentiments were often tempered by fears that the mass of the population was \uline{~~20~~} self-rule and democracy.