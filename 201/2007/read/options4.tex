\item The statement ``It never rains but it pours'' is used to introduce
\begin{tasks}
	\task the fierce business competition.
	\task the feeble boss-board relations.
	\task the threat from news reports.
	\task the severity of data leakage.
\end{tasks}
\item According to Paragraph 2, some organizations check their systems to find out
\begin{tasks}
	\task whether there is any weak point.
	\task what sort of data has been stolen.
	\task who is responsible for the leakage.
	\task how the potential spies can be located.
\end{tasks}
\item In bringing up the concept of GASP the author is making the point that
\begin{tasks}
	\task shareholders' interests should be properly attended to.
	\task information protection should be given due attention.
	\task businesses should enhance their level of accounting security.
	\task the market value of customer data should be emphasized.
\end{tasks}
\item According to Paragraph 4, what puzzles the author is that some bosses fail to
\begin{tasks}
	\task see the link between trust and data protection.
	\task perceive the sensitivity of personal data.
	\task realize the high cost of data restoration.
	\task appreciate the economic value of trust.
\end{tasks}
\item It can be inferred from Paragraph 5 that
\begin{tasks}
	\task data leakage is more severe in Europe.
	\task FTC's decision is essential to data security.
	\task California takes the lead in security legislation.
	\task legal penalty is a major solution to data leakage.
\end{tasks}