\item Which of the following may be required in an intelligence test?
\begin{tasks}
	\task Answering philosophical questions.
	\task Folding or cutting paper into different shapes.
	\task Telling the differences between certain concepts.
	\task Choosing words or graphs similar to the given ones.
\end{tasks}
\item What can be inferred about intelligence testing from Paragraph 3?
\begin{tasks}
	\task People no longer use IQ scores as an indicator of intelligence.
	\task More versions of IQ tests are now available on the Internet.
	\task The test contents and formats for adults and children may be different.
	\task Scientists have defined the important elements of human intelligence.
\end{tasks}
\item People nowadays can no longer achieve IQ scores as high as vos Savant' s because
\begin{tasks}
	\task the scores are obtained through different computational procedures.
	\task creativity rather than analytical skills is emphasized now.
	\task vos Savant' s case is an extreme one that will not repeat.
	\task the defining characteristic of IQ tests has changed.
\end{tasks}
\item We can conclude from the last paragraph that
\begin{tasks}
	\task test scores may not be reliable indicators of one' s ability.
	\task IQ scores and SAT results are highly correlated.
	\task testing involves a lot of guesswork.
	\task traditional tests are out of date.
\end{tasks}
\item What is the author' s attitude towards IQ tests?
\begin{tasks}
	\task Supportive.
	\task Skeptical.
	\task Impartial.
	\task Biased.
\end{tasks}