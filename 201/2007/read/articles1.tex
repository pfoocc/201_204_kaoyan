If you were to examine the birth certificates of every soccer player in 2006's World Cup tournament, you would most likely find a noteworthy quirk: elite soccer players are more likely to have been born in the earlier months of the year than in the later months. If you then examined the European national youth teams that feed the World Cup and professional ranks, you would find this strange phenomenon to be even more pronounced.


What might account for this strange phenomenon? Here are a few guesses: a) certain astrological signs confer superior soccer skills; b) winter-born babies tend to have higher oxygen capacity, which increases soccer stamina; c) soccer-mad parents are more likely to conceive children in springtime, at the annual peak of soccer mania;d) none of the above.


Anders Ericsson, a 58-year-old psychology professor at Florida State University, says he believes strongly in ``none of the above.'' Ericsson grew up in Sweden, and studied nuclear engineering until he realized he would have more opportunity to conduct his own research if he switched to psychology. His first experiment, nearly 30 years ago, involved memory: training a person to hear and then repeat a random series of numbers. ``With the first subject, after about 20 hours of training, his digit span had risen from 7 to 20,'' Ericsson recalls. ``He kept improving, and after about 200 hours of training he had risen to over 80 numbers.''


This success, coupled with later research showing that memory itself is not genetically determined, led Ericsson to conclude that the act of memorizing is more of a cognitive exercise than an intuitive one. In other words, whatever inborn differences two people may exhibit in their abilities to memorize, those differences are swamped by how well each person ``encodes'' the information. And the best way to learn how to encode information meaningfully, Ericsson determined, was a process known as deliberate practice. Deliberate practice entails more than simply repeating a task. Rather, it involves setting specific goals, obtaining immediate feedback and concentrating as much on technique as on outcome.


Ericsson and his colleagues have thus taken to studying expert performers in a wide range of pursuits, including soccer. They gather all the data they can, not just performance statistics and biographical details but also the results of their own laboratory experiments with high achievers. Their work makes a rather startling assertion: the trait we commonly call talent is highly overrated. Or, put another way, expert performers – whether in memory or surgery, ballet or computer programming – are nearly always made, not born.