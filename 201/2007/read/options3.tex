\item Today' s double-income families are at greater financial risk in that
\begin{tasks}
	\task the safety net they used to enjoy has disappeared.
	\task their chances of being laid off have greatly increased.
	\task they are more vulnerable to changes in family economics.
	\task they are deprived of unemployment or disability insurance.
\end{tasks}
\item As a result of President Bush' s reform, retired people may have
\begin{tasks}
	\task a higher sense of security.
	\task less secured payments.
	\task less chance to invest.
	\task a guaranteed future.
\end{tasks}
\item According to the author, health-savings plans will
\begin{tasks}
	\task help reduce the cost of healthcare.
	\task popularize among the middle class.
	\task compensate for the reduced pensions.
	\task increase the families' investment risk.
\end{tasks}
\item It can be inferred from the last paragraph that
\begin{tasks}
	\task financial risks tend to outweigh political risks.
	\task the middle class may face greater political challenges.
	\task financial problems may bring about political problems.
	\task financial responsibility is an indicator of political status.
\end{tasks}
\item Which of the following is the best title for this text?
\begin{tasks}
	\task The Middle Class on the Alert
	\task The Middle Class on the Cliff
	\task The Middle Class in Conflict
	\task The Middle Class in Ruins
\end{tasks}