It is a wise father that knows his own child, but today a man can boost his paternal (fatherly) wisdom – or at least confirm that he's the kid's dad. All he needs to do is shell out \$30 for a paternity testing kit (PTK) at his local drugstore – and another \$120 to get the results.


More than 60, 000 people have purchased the PTKs since they first became available without prescriptions last year, according to Doug Fogg, chief operating officer of Identigene, which makes the over-the-counter kits. More than two dozen companies sell DNA tests directly to the public, ranging in price from a few hundred dollars to more than \$ 2, 500.


Among the most popular: paternity and kinship testing, which adopted children can use to find their biological relatives and families can use to track down kids put up for adoption. DNA testing is also the latest rage among passionate genealogists – and supports businesses that offer to search for a family's geographic roots.


Most tests require collecting cells by swabbing saliva in the mouth and sending it to the company for testing. All tests require a potential candidate with whom to compare DNA.


But some observers are skeptical. ``There's a kind of false precision being hawked by people claiming they are doing ancestry testing,'' says Troy Duster, a New York University sociologist. He notes that each individual has many ancestors – numbering in the hundreds just a few centuries back. Yet most ancestry testing only considers a single lineage, either the Y chromosome inherited through men in a father's line or mitochondrial DNA, which is passed down only from mothers. This DNA can reveal genetic information about only one or two ancestors, even though, for example, just three generations back people also have six other great-grandparents or, four generations back, 14 other great-great-grandparents.


Critics also argue that commercial genetic testing is only as good as the reference collections to which a sample is compared. Databases used by some companies don't rely on data collected systematically but rather lump together information from different research projects. This means that a DNA database may have a lot of data from some regions and not others, so a person's test results may differ depending on the company that processes the results. In addition, the computer programs a company uses to estimate relationships may be patented and not subject to peer review or outside evaluation.