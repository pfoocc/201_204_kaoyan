\item The author holds in Paragraph 1 that the importance of education in poor countries
\begin{tasks}
	\task is subject to groundless doubts.
	\task has fallen victim of bias.
	\task is conventionally downgraded.
	\task has been overestimated.
\end{tasks}
\item It is stated in Paragraph 1 that the construction of a new educational system
\begin{tasks}
	\task challenges economists and politicians.
	\task takes efforts of generations.
	\task demands priority from the government.
	\task requires sufficient labor force.
\end{tasks}
\item A major difference between the Japanese and U.S. workforces is that
\begin{tasks}
	\task the Japanese workforce is better disciplined.
	\task the Japanese workforce is more productive.
	\task the U.S. workforce has a better education.
	\task the U.S. workforce is more organized.
\end{tasks}
\item The author quotes the example of our ancestors to show that education emerged
\begin{tasks}
	\task when people had enough time.
	\task prior to better ways of finding food.
	\task when people no longer went hungry.
	\task as a result of pressure on government.
\end{tasks}
\item According to the last paragraph, development of education
\begin{tasks}
	\task results directly from competitive environments.
	\task does not depend on economic performance.
	\task follows improved productivity.
	\task cannot afford political changes.
\end{tasks}