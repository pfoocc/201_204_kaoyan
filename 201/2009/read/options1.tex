\item In Wordsworth's view, ``habits'' is characterized by being
\begin{tasks}
	\task casual.
	\task familiar.
	\task mechanical.
	\task changeable.
\end{tasks}
\item Brain researchers have discovered that the formation of new habits can be
\begin{tasks}
	\task predicted.
	\task regulated.
	\task traced.
	\task guided.
\end{tasks}
\item The word ``ruts'' (Para. 4) is closest in meaning to
\begin{tasks}
	\task tracks.
	\task series.
	\task characteristics.
	\task connections.
\end{tasks}
\item Dawna Markova would most probably agree that
\begin{tasks}
	\task ideas are born of a relaxing mind.
	\task innovativeness could be taught.
	\task decisiveness derives from fantastic ideas.
	\task curiosity activates creative minds.
\end{tasks}
\item Ryan's comments suggest that the practice of standardized testing
\begin{tasks}
	\task prevents new habits from being formed.
	\task no longer emphasizes commonness.
	\task maintains the inherent American thinking mode.
	\task complies with the American belief system.
\end{tasks}