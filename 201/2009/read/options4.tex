\item The author holds that in the seventeenth-century New England
\begin{tasks}
	\task Puritan tradition dominated political life.
	\task intellectual interests were encouraged.
	\task politics benefited much from intellectual endeavors.
	\task intellectual pursuits enjoyed a liberal environment.
\end{tasks}
\item It is suggested in Paragraph 2 that New Englanders
\begin{tasks}
	\task experienced a comparatively peaceful early history.
	\task brought with them the culture of the Old World.
	\task paid little attention to southern intellectual life.
	\task were obsessed with religious innovations.
\end{tasks}
\item The early ministers and political leaders in Massachusetts Bay
\begin{tasks}
	\task were famous in the New World for their writings.
	\task gained increasing importance in religious affairs.
	\task abandoned high positions before coming to the New World.
	\task created a new intellectual atmosphere in New England.
\end{tasks}
\item The story of John Dane shows that less well-educated New Englanders were  often
\begin{tasks}
	\task influenced by superstitions.
	\task troubled with religious beliefs.
	\task puzzled by church sermons.
	\task frustrated with family earnings.
\end{tasks}
\item The text suggests that early settlers in New England
\begin{tasks}
	\task were mostly engaged in political activities.
	\task were motivated by an illusory prospect.
	\task came from different intellectual backgrounds.
	\task left few formal records for later reference.
\end{tasks}