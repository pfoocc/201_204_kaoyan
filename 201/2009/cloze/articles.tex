Research on animal intelligence always makes us wonder just how smart humans are. \uline{~~1~~} the fruit-fly experiments described by Carl Zimmer in the  Science Times. Fruit flies who were taught to be smarter than the average fruit fly \uline{~~2~~} to live shorter lives. This suggests that \uline{~~3~~} bulbs burn longer, that  there is a(n) \uline{~~4~~} in not being too bright.


Intelligence, it \uline{~~5~~}, is a high-priced option. It takes more upkeep, burns more fuel and is slow \uline{~~6~~} the starting line because it depends on learning – a(n)


\uline{~~7~~} process – instead of instinct. Plenty of other species are able to learn, and one of the things they've apparently learned is when to \uline{~~8~~}.


Is there an adaptive value to \uline{~~9~~} intelligence? That's the question behind this new research. Instead of casting a wistful glance \uline{~~10~~} at all the species we've left in the dust I.Q.-wise, it implicitly asks what the real \uline{~~11~~} of our own intelligence might be. This is \uline{~~12~~} the mind of every animal we've ever met.


Research on animal intelligence also makes us wonder what experiments animals would \uline{~~13~~} on humans if they had the chance. Every cat with an owner, \uline{~~14~~}, is running a small-scale study in operant conditioning. We believe that \uline{~~15~~} animals ran the labs, they would test us to \uline{~~16~~} the limits of our patience, our faithfulness, our memory for locations. They would try to decide what intelligence in humans is really \uline{~~17~~}, not merely how much of it there is. \uline{~~18~~}, they would hope to   study a(n) \uline{~~19~~} question: Are humans actually aware of the world they live  in? \uline{~~20~~} the results are inconclusive.


