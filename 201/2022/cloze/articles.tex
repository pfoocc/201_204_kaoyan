The idea that plants have some degree of consciousness first took root in the early 2000s; the term ``plant neurobiology'' was \uline{~~1~~} around the notion that some aspects of plant behavior could be \uline{~~2~~} to intelligence in animals. \uline{~~3~~} plants lack brains, the firing of electrical signals in their stems and leaves nonetheless triggered responses that \uline{~~4~~} consciousness, researchers previously reported.


But such an idea is untrue, according to a new opinion article. Plant biology is complex and fascinating, but it \uline{~~5~~} so greatly from that of animals that so-called \uline{~~6~~} of plants' intelligence is inconclusive, the authors wrote.


Beginning in 2006, some scientists have \uline{~~7~~} that plants possess neuron-like cells that interact with hormones and neurotransmitters, \uline{~~8~~} ``a plant nervous system, \uline{~~9~~} to that in animals, ''said lead study author Lincoln Taiz, ``They \uline{~~10~~} claimed that plants have 'brain-like command centers' at their root tips.''


This \uline{~~11~~} makes sense if you simplify the workings of a complex brain, \uline{~~12~~} it to an array of electrical pulses; cells in plants also communicate through electrical signals. \uline{~~13~~} , the signaling in a plant is only \uline{~~14~~} similar to the firing in a complex animal brain, which is more than ``a mass of cells that communicate by electricity,'' Taiz said.


``For consciousness to evolve, a brain with a threshold \uline{~~15~~} of complexity and capacity is required,'' he \uline{~~16~~} . ``Since plants don't have nervous systems, the \uline{~~17~~} that they have consciousness are effectively zero.''


And what's so great about consciousness, anyway? Plants can't run away from \uline{~~18~~} , so investing energy in a body system which \uline{~~19~~} a threat and can feel pain would be a very \uline{~~20~~} evolutionary strategy, according to the article.