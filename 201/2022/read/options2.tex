\item the author suggests that Generation Z should \uline{~~~~}.
\begin{tasks}
	\task be careful in choosing a college
	\task be diligent at each educational stage
	\task reassess the necessity of college education
	\task postpone their undergraduate application
\end{tasks}
\item The percentage of UK graduates in non-graduate roles reflect \uline{~~~~}.
\begin{tasks}
	\task Millennial's opinions about work
	\task the shrinking value of a degree
	\task public discontent with education
	\task the desired route of social mobility
\end{tasks}
\item The author considers it a good sign that \uline{~~~~}.
\begin{tasks}
	\task Generation Z are seeking to earn a decent degree.
	\task school leavers are willing to be skilled workers.
	\task employers are taking a realistic attitude to degrees.
	\task parents are changing their minds about education.
\end{tasks}
\item It is advised in Paragraph 5 that those with one degree should \uline{~~~~}.
\begin{tasks}
	\task make an early decision on their career
	\task attend on the job training programs
	\task team up with high-paid postgraduates
	\task further their studies in a specific field
\end{tasks}
\item What can be concluded about Generation Z from the last two paragraphs?
\begin{tasks}
	\task Lifelong learning will define them.
	\task They will make qualified educators.
	\task Degrees will no longer appeal them.
	\task They will have a limited choice of jobs.
\end{tasks}
