Caravanserais were roadside inns that were built along the Silk Road in areas including China, North Africa and the Middle East. They were typically \uline{~~1~~} outside the walls of a city or village and were usually funded by governments of \uline{~~2~~} .


This word ``Caravanserais'' is a \uline{~~3~~} of the Persian word ``karvan'', which means a group of travellers or a caravan, and seray, a palace or enclosed building. The Perm caravan was used to \uline{~~4~~} groups of people who travelled together across the ancient network for safety reasons, \uline{~~5~~} merchants, travellers or pilgrims.


From the 10th century onwards, as merchant and travel routes become more developed, the \uline{~~6~~} of the Caravanserais increased and they served as a safe place for people to rest at night. Travellers on the Silk Road \uline{~~7~~} possibility of being attacked by thieves or being \uline{~~8~~} to extreme conditions. For this reason, Caravanserais were strategically placed \uline{~~9~~} they could be reached in a day's travel time.


Caravanserais served as an informal \uline{~~10~~} point for the various people who travelled the Silk Road. \uline{~~11~~} , those structures became important centers for culture \uline{~~12~~} and interaction, with travelers sharing their cultures, ideas and beliefs, \uline{~~13~~} talking knowledge with them, greatly \uline{~~14~~} the development of several civilizations.


Caravanserais were also an important marketplace for commodities and \uline{~~15~~} in the trade of goods along the Silk Road. \uline{~~16~~} , it was frequently the first stop merchants looking to sell their wares and \uline{~~17~~} supplies for their own journeys. It is \uline{~~18~~} that around 120000 to 15000 caravanserais were built along the Silk Road, \uline{~~19~~} only about 3000 are known to remain today, many of which are in \uline{~~20~~} .