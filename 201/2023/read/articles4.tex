Scientific papers are the recordkeepers of progress in research. Each year researchers publish millions of papers in more than 30,000 journals. The scientific community measures the quality of those papers in a number of ways, including the perceived quality of the journal (as reflected by the title's impact factor) and the number of citations a specific paper accumulates. The careers of scientists and the reputation of their institutions depend on the number and prestige of the papers they produce, but even more so on the citations attracted by these papers.


In recent years, there have been several episodes of scientific fraud, including completely made-up data, massaged or doctored figures, multiple publications of the same data, theft of complete articles, plagiarism of text, and self-plagiarism. And some scientists have come up with another way to artificially boost the number of citations to their work.


Citation cartels, where journals, authors, and institutions conspire to inflate citation numbers, have existed for a long time. In 2016, researchers developed an algorithm to recognize suspicious citation patterns, including groups of authors that disproportionately cite one another and groups of journals that cite each other frequently to increase the impact factors of their publications. Recently, I came across yet another expression of this predatory behavior: so-called support service consultancies that provide language and other editorial support to individual authors and to journals sometimes advise contributors to add a number of citations to their articles and the articles of colleagues. Some of these consultancies are also active in organizing conferences and can advise that citations be added to conference proceedings. In this manner, a single editor can drive hundreds of citations in the direction of his own articles or those of colleagues that may be in his circle.


How insidious is this type of citation manipulation? In one example, an individual—acting as author, editor, and consultant—was able to use at least 15 journals as citation providers to articles published by five scientists at three universities. The problem is rampant in Scopus, which includes a high number of the new ``international'' journals. In fact, a listing in Scopus seems to be a criterion to be targeted in this type of citation manipulation.