Communities throughout the region have been attempting to regulate short-term rentals since sites like Airbnb took off in the 2010s. Now, with record-high home prices and historically low inventory, there's an increased urgency in such regulation, particularly among those who worry that developers will come in and buy up swaths of housing to flip for a fortune on the short-term rental market.


In New Hampshire, where the rental vacancy rate has dropped below 1 percent, housing advocates fear unchecked short-term rentals will put further pressure on an already strained market. The state Legislature recently voted against a bill that would've made it illegal for towns to create legislation restricting short-term rentals.


``We are at a crisis level on the supply of rental housing, so anytime you're taking the tool out of the toolkit for communities to address this, you're potentially taking supply off the market that's already incredibly stressed,'' said Nick Taylor, executive director of the Workforce Housing Coalition of the Greater Seacoast. Without enough affordable housing in southern New Hampshire towns, ``employers are having a hard time attracting employees, and workers are having a hard time finding a place to live,'' Taylor said.


However, short-term rentals also provide housing for tourists, a crucial part of the economies in places like Nantucket, Cape Cod, or the towns that make up New Hampshire's Seacoast and Lakes Region, pointed out Ryan Castle, CEO of the Cape Cod \& Islands Association of Realtors. ``A lot of workers are servicing the tourist industry, and the tourism industry is serviced by those people coming in short term,'' Castle said, ``and so it's a cyclical effect.''


Short-term rentals themselves are not the crux of the issue, said Keren Horn, an affordable housing policy expert at the University of Massachusetts Boston. ``I think individuals being able to rent out their second home is a good thing. If it's their vacation home anyway, and it's just empty, why can't you make money off it?'' Horn said. Issues arise, however, when developers attempt to create large-scale short-term rental facilities — de facto hotels — to bypass taxes and regulations. ``I think the question is, shouldn't a developer who's really building a hotel, but disguising it as not a hotel, be treated and taxed and regulated like a hotel?'' Horn said.


At the end of 2018, Governor Charlie Baker signed a bill to rein in those potential investor-buyers. ``The bill requires every rental host to register with the state, mandates they carry insurance, and opens the potential for local taxes on top of a new state levy,'' the Globe reported. Boston took things even further, limiting who is authorized to rent out their home, and requiring renters to register with the city's Inspectional Services Department.


Horn said similar registration requirements could benefit other struggling cities and towns. The only way to solve the issue, however, is by creating more housing. ``If we want to make a change in the housing market, the main one is we have to build a lot more.''