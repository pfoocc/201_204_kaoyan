The weather in Texas may have cooled since the recent extreme heat, but the temperature will be high at the State Board of Education meeting in Austin this month as officials debate how climate change is taught in Texas schools.


Pat Hardy, who sympathized with views of the energy sector, is resisting the proposed change to science standards for pre-teen pupils. These would emphasise the primacy of human activity in recent climate change and encourage discussion of mitigation measures.


Most scientists and experts sharply dispute Hardy's views. ``They casually dismiss the career work of scholars and scientists as just another misguided opinion.'' says Dan Quinn, senior communications strategist at the Texas Freedom Network, a non-profit group that monitors public education, ``What millions of Texas kids learn in their public schools is determined too often by the political ideology of partisan board members, rather than facts and sound scholarship.''


Such debate reflects fierce discussion discussions across the US and around the world, as researchers, policymakers, teachers and students step up demands for a greater focus on teaching about the facts of climate change in schools.


A study last year by the National Center for Science Education, a non-profit group of scientists and teachers, looking at how state public schools across the country address climate change in science classes, gave barely half of US states a grade B+ or higher. Among the 10 worst performers were some of the most populous states, including Texas, which was given the lowest grade (F) and has a disproportionate influence because its textbooks are widely sold elsewhere.


Glenn Branch, the centre's deputy director, cautions that setting state-level science standards is only one limited benchmark in a country that decentralises decisions to local school boards. Even if a state is considered a high performer in its science standards, ``that does not mean it will be taught'', he says.


Another issue is that while climate change is well integrated into some subjects and at some ages — such as earth and space sciences in high schools — it is not as well represented in curricula for younger children and in subjects that are more widely taught, such as biology and chemistry. It is also less prominent in many social studies courses.


Branch points out that, even if a growing number of official guidelines and textbooks reflect scientific consensus on climate change, unofficial educational materials that convey more slanted perspectives are being distributed to teachers. They include materials sponsored by libertarian think-tanks and energy industry associations.