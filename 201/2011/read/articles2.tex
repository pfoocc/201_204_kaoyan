When Liam McGee departed as president of Bank of America in August, his explanation was surprisingly straight up. Rather than cloaking his exit in the usual vague excuses, he came right out and said he was leaving ``to pursue my goal of running a company.'' Broadcasting his ambition was ``very much my decision,'' McGee says. Within two weeks, he was talking for the first time with the board of Hartford Financial Services Group, which named him CEO and chairman on September 29.


McGee says leaving without a position lined up gave him time to reflect on what kind of company he wanted to run. It also sent a clear message to the outside world about his aspirations. And McGee isn't alone. In recent weeks the No.2 executives at Avon and American Express quit with the explanation that they were looking for a CEO post. As boards scrutinize succession plans in response to shareholder pressure, executives who don't get the nod also may wish to move on. A turbulent business environment also has senior managers cautious of letting vague pronouncements cloud their reputations.


As the first signs of recovery begin to take hold, deputy chiefs may be more willing to make the jump without a net. In the third quarter, CEO turnover was down 23\% from a year ago as nervous boards stuck with the leaders they had, according to Liberum Research. As the economy picks up, opportunities will abound for aspiring leaders.


The decision to quit a senior position to look for a better one is unconventional. For years executives and headhunters have adhered to the rule that the most attractive CEO candidates are the ones who must be poached. Says Korn/Ferry senior partner Dennis Carey: ``I can't think of a single search I've done where a board has not instructed me to look at sitting CEOs first.''


Those who jumped without a job haven't always landed in top positions quickly. Ellen Marram quit as chief of Tropicana a decade ago, saying she wanted to be a CEO. It was a year before she became head of a tiny Internet-based commodities exchange. Robert Willumstad left Citigroup in 2005 with ambitions to be a CEO. He finally took that post at a major financial institution three years later.


Many recruiters say the old disgrace is fading for top performers. The financial crisis has made it more acceptable to be between jobs or to leave a bad one. ``The traditional rule was it's safer to stay where you are, but that's been fundamentally inverted,'' says one headhunter. ``The people who've been hurt the worst are those who've stayed too long.''