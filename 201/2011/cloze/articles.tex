Ancient Greek philosopher Aristotle viewed laughter as ``a bodily exercise precious to health.'' But \uline{~~1~~} some claims to the contrary, laughing probably has  little influence on physical fitness. Laughter does \uline{~~2~~} short-term changes in the function of the heart and its blood vessels, \uline{~~3~~} heart rate and oxygen consumption. But because hard laughter is difficult to \uline{~~4~~}, a good laugh is  unlikely to have \uline{~~5~~} benefits the way, say, walking or jogging does.


\uline{~~6~~}, instead of straining muscles to build them, as exercise does, laughter apparently accomplishes the \uline{~~7~~}. Studies dating back to the 1930s indicate that laughter \uline{~~8~~} muscles, decreasing muscle tone for up to 45 minutes after the laugh dies down.


Such bodily reaction might conceivably help \uline{~~9~~} the effects of psychological stress. Anyway, the act of laughing probably does produce other types of \uline{~~10~~} feedback that improve an individual's emotional state. \uline{~~11~~} one classical theory of emotion, our feelings are partially rooted \uline{~~12~~} physical reactions. It was argued at the end of the 19th century that humans do not cry \uline{~~13~~} they are sad but they become sad when the tears begin to flow.


Although sadness also \uline{~~14~~} tears, evidence suggests that emotions can flow \uline{~~15~~} muscular responses. In an experiment published in 1988, social psychologist Fritz Strack of the University of Würzburg in Germany asked volunteers to \uline{~~16~~} a pen either with their teeth – thereby creating an artificial smile – or with their lips, which would produce a(n) \uline{~~17~~} expression. Those forced to exercise  their smiling muscles \uline{~~18~~} more enthusiastically to funny cartoons than did those whose mouths were contracted in a frown, \uline{~~19~~} that expressions may influence emotions rather than just the other way around. \uline{~~20~~}, the physical act of laughter could improve mood.