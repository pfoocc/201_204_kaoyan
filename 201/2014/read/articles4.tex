``The Heart of the Matter,'' the just-released report by the American Academy of Arts and Sciences (AAAS), deserves praise for affirming the importance of the humanities and social sciences to the prosperity and security of liberal democracy   in America. Regrettably, however, the report's failure to address the true nature of the crisis facing liberal education may cause more harm than good.


In 2010, leading congressional Democrats and Republicans sent letters to the AAAS asking that it identify actions that could be taken by ``federal, state and local governments, universities, foundations, educators, individual benefactors and others'' to ``maintain national excellence in humanities and social scientific scholarship and education.'' In response, the American Academy formed the Commission on the Humanities and Social Sciences. Among the commission's 51 members are top-tier-university presidents, scholars, lawyers, judges, and business executives, as well as prominent figures from diplomacy, filmmaking, music and journalism.


The goals identified in the report are generally admirable. Because representative government presupposes an informed citizenry, the report supports full literacy; stresses the study of history and government, particularly American history and American government; and encourages the use of new digital technologies. To encourage innovation and competition, the report calls for increased investment in research, the crafting of coherent curricula that improve students' ability to solve problems and communicate effectively in the 21st century, increased funding for teachers and the encouragement of scholars to bring their learning to bear on the great challenges of the day. The report also advocates greater study of foreign languages, international affairs and the expansion of study abroad programs.


Unfortunately, despite 21/2 years in the making, ``The Heart of the Matter'' never gets to the heart of the matter: the illiberal nature of liberal education at our leading colleges and universities. The commission ignores that for several decades America's colleges and universities have produced graduates who don't know the content and character of liberal education and are thus deprived of its benefits. Sadly, the spirit of inquiry once at home on campus has been replaced by the use  of the humanities and social sciences as vehicles for publicizing ``progressive,'' or left-liberal propaganda.


Today, professors routinely treat the progressive interpretation of history and progressive public policy as the proper subject of study while portraying conservative or classical liberal ideas – such as free markets and self-reliance – as falling outside the boundaries of routine, and sometimes legitimate, intellectual investigation.


The AAAS displays great enthusiasm for liberal education. Yet its report may well set back reform by obscuring the depth and breadth of the challenge that Congress asked it to illuminate.