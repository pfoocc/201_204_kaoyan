All around the world, lawyers generate more hostility than the members of any other profession – with the possible exception of journalism. But there are few places where clients have more grounds for complaint than America.


During the decade before the economic crisis, spending on legal services in America grew twice as fast as inflation. The best lawyers made skyscrapers-full of money, tempting ever more students to pile into law schools. But most law graduates never get a big-firm job. Many of them instead become the kind of nuisance-lawsuit filer that makes the tort system a costly nightmare.


There are many reasons for this. One is the excessive costs of a legal education. There is just one path for a lawyer in most American states: a four-year undergraduate degree in some unrelated subject, then a three-year law degree at one of 200 law schools authorized by the American Bar Association and an expensive preparation for the bar exam. This leaves today's average law-school graduate with \$100,000 of debt on top of undergraduate debts. Law-school debt means that they have to work fearsomely hard.


Reforming the system would help both lawyers and their customers. Sensible ideas have been around for a long time, but the state-level bodies that govern the profession have been too conservative to implement them. One idea is to allow people to study law as an undergraduate degree. Another is to let students sit for the bar after only two years of law school. If the bar exam is truly a stern enough test for a would-be lawyer, those who can sit it earlier should be allowed to do so. Students who do not need the extra training could cut their debt mountain by a third.


The other reason why costs are so high is the restrictive guild-like ownership structure of the business. Except in the District of Columbia, non-lawyers may not own any share of a law firm. This keeps fees high and innovation slow. There is pressure for change from within the profession, but opponents of change among the regulators insist that keeping outsiders out of a law firm isolates lawyers from the pressure to make money rather than serve clients ethically.


In fact, allowing non-lawyers to own shares in law firms would reduce costs  and improve services to customers, by encouraging law firms to use technology and to employ professional managers to focus on improving firms' efficiency. After all, other countries, such as Australia and Britain, have started liberalizing their legal professions. America should follow.


