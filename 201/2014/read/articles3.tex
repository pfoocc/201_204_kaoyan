The US\$3-million Fundamental Physics Prize is indeed an interesting experiment, as Alexander Polyakov said when he accepted this year's award in March. And it is far from the only one of its type. As a News Feature article in Nature discusses, a string of lucrative awards for researchers have joined the Nobel Prizes in recent years. Many, like the Fundamental Physics Prize, are funded from the telephone-number-sized bank accounts of Internet entrepreneurs. These benefactors have succeeded in their chosen fields, they say, and they want to use their wealth to draw attention to those who have succeeded in science.


What's not to like? Quite a lot, according to a handful of scientists quoted in the News Feature. You cannot buy class, as the old saying goes, and these upstart entrepreneurs cannot buy their prizes the prestige of the Nobels. The new awards  are an exercise in self-promotion for those behind them, say scientists. They could distort the achievement-based system of peer-review-led research. They could cement the status quo of peer-reviewed research. They do not fund peer-reviewed research. They perpetuate the myth of the lone genius.


The goals of the prize-givers seem as scattered as the criticism. Some want to shock, others to draw people into science, or to better reward those who have made their careers in research.


As Nature has pointed out before, there are some legitimate concerns about how science prizes – both new and old – are distributed. The Breakthrough Prize in Life Sciences, launched this year, takes an unrepresentative view of what the life sciences include. But the Nobel Foundation's limit of three recipients per prize, each of whom must still be living, has long been outgrown by the collaborative nature of modern research – as will be demonstrated by the inevitable row over who is ignored when it comes to acknowledging the discovery of the Higgs boson. The Nobels were, of course, themselves set up by a very rich individual who had decided what he wanted to do with his own money. Time, rather than intention, has given them legitimacy.


As much as some scientists may complain about the new awards, two things seem clear. First, most researchers would accept such a prize if they were offered one. Second, it is surely a good thing that the money and attention come to science rather than go elsewhere. It is fair to criticize and question the mechanism– that is the culture of research, after all – but it is the prize-givers' money to do with as they please. It is wise to take such gifts with gratitude and grace.