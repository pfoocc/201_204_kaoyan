\item George Osborne's scheme was intended to
\begin{tasks}
	\task encourage jobseekers' active engagement in job seeking.
	\task provide the unemployed with easier access to benefits.
	\task guarantee jobseekers' legitimate right to benefits.
	\task motivate the unemployed to report voluntarily.
\end{tasks}
\item The phrase ``to sign on'' (Line 3, Para. 2) most probably means
\begin{tasks}
	\task to check on the availability of jobs at the jobcentre.
	\task to accept the government's restrictions on the allowance.
	\task to register for an allowance from the government.
	\task to attend a governmental job-training program.
\end{tasks}
\item What prompted the chancellor to develop his scheme?
\begin{tasks}
	\task A desire to secure a better life for all.
	\task An eagerness to protect the unemployed.
	\task An urge to be generous to the claimants.
	\task A passion to ensure fairness for taxpayers.
\end{tasks}
\item According to Paragraph 3, being unemployed makes one feel
\begin{tasks}
	\task uneasy.
	\task insulted.
	\task enraged.
	\task guilty.
\end{tasks}
\item To which of the following would the author most probably agree?
\begin{tasks}
	\task Unemployment benefits should not be made conditional.
	\task The British welfare system indulges jobseekers' laziness.
	\task The jobseekers' allowance has met their actual needs.
	\task Osborne's reforms will reduce the risk of unemployment.
\end{tasks}