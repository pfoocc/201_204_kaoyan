The homeless make up a growing percentage of America's population. \uline{~~1~~}, homelessness has reached such proportions that local governments can't possibly \uline{~~2~~}. To help homeless people \uline{~~3~~} independence, the federal government must support job training programs, \uline{~~4~~} the minimum wage, and fund more low-cost housing.


\uline{~~5~~} everyone agrees on the number of Americans who are homeless. Estimates \uline{~~6~~} anywhere from 600,000 to 3 million. \uline{~~7~~} the figure may vary, analysts do agree on another matter: that the number of the homeless is \uline{~~8~~}. One of the federal government's studies \uline{~~9~~} that the number of the homeless will reach nearly 19 million by the end of this decade.


Finding ways to \uline{~~10~~} this growing homeless population has become increasingly difficult. \uline{~~11~~} when homeless individuals manage to find a \uline{~~12~~} that will give them three meals a day and a place to sleep at night, a good number still spend the bulk of each day \uline{~~13~~} the street. Part of the problem is that many homeless adults  are addicted to alcohol or drugs. And a significant number of the homeless have serious mental disorders. Many others, \uline{~~14~~} not addicted or mentally ill, simply lack the everyday \uline{~~15~~} skills needed to turn their lives \uline{~~16~~}. Boston Globe reporter Chris Reidy notes that the situation will improve only when there are \uline{~~17~~} programs that address the many needs of the homeless.


\uline{~~18~~} Edward Zlotkowski, director of community service at Bentley College in Massachusetts, \uline{~~19~~} it, ``There has to be \uline{~~20~~} of programs. What's needed is a package deal.''