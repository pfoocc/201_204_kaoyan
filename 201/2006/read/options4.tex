\item By citing the examples of poets Wordsworth and Baudelaire, the author intends   to show that
\begin{tasks}
	\task poetry is not as expressive of joy as painting or music.
	\task art grows out of both positive and negative feelings.
	\task poets today are less skeptical of happiness.
	\task artists have changed their focus of interest.
\end{tasks}
\item The word ``bummer'' (Line 5, Paragraph 5) most probably means something
\begin{tasks}
	\task religious.
	\task unpleasant.
	\task entertaining.
	\task commercial.
\end{tasks}
\item In the author's opinion, advertising
\begin{tasks}
	\task emerges in the wake of the anti-happy art.
	\task is a cause of disappointment for the general public.
	\task replaces the church as a major source of information.
	\task creates an illusion of happiness rather than happiness itself.
\end{tasks}
\item We can learn from the last paragraph that the author believes
\begin{tasks}
	\task happiness more often than not ends in sadness.
	\task the anti-happy art is distasteful but refreshing.
	\task misery should be enjoyed rather than denied.
	\task the anti-happy art flourishes when economy booms.
\end{tasks}
\item Which of the following is true of the text?
\begin{tasks}
	\task Religion once functioned as a reminder of misery.
	\task Art provides a balance between expectation and reality.
	\task People feel disappointed at the realities of modern society.
	\task Mass media are inclined to cover disasters and deaths.
\end{tasks}