\item The extinction of large prehistoric animals is noted to suggest that
\begin{tasks}
	\task large animals were vulnerable to the changing environment.
	\task small species survived as large animals disappeared.
	\task large sea animals may face the same threat today.
	\task slow-growing fish outlive fast-growing ones.
\end{tasks}
\item We can infer from Dr. Myers and Dr. Worm's paper that
\begin{tasks}
	\task the stock of large predators in some old fisheries has reduced by 90\%.
	\task there are only half as many fisheries as there were 15 years ago.
	\task the catch sizes in new fisheries are only 20\% of the original amount.
	\task the number of large predators dropped faster in new fisheries than in the old.
\end{tasks}
\item By saying ``these figures are conservative'' (Line 1, Paragraph 3), Dr. Worm means that
\begin{tasks}
	\task fishing technology has improved rapidly.
	\task the catch-sizes are actually smaller than recorded.
	\task the marine biomass has suffered a greater loss.
	\task the data collected so far are out of date.
\end{tasks}
\item Dr. Myers and other researchers hold that
\begin{tasks}
	\task people should look for a baseline that can work for a longer time.
	\task fisheries should keep their yields below 50\% of the biomass.
	\task the ocean biomass should be restored to its original level.
	\task people should adjust the fishing baseline to the changing situation.
\end{tasks}
\item The author seems to be mainly concerned with most fisheries'
\begin{tasks}
	\task management efficiency.
	\task biomass level.
	\task catch-size limits.
	\task technological application.
\end{tasks}