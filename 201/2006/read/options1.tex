\item The word ``homogenizing'' (Line 2, Paragraph 1) most probably means
\begin{tasks}
	\task identifying.
	\task associating.
	\task assimilating.
	\task monopolizing.
\end{tasks}
\item According to the author, the department stores of the 19th century
\begin{tasks}
	\task played a role in the spread of popular culture.
	\task became intimate shops for common consumers.
	\task satisfied the needs of a knowledgeable elite.
	\task owed its emergence to the culture of consumption.
\end{tasks}
\item The text suggests that immigrants now in the U.S.
\begin{tasks}
	\task are resistant to homogenization.
	\task exert a great influence on American culture.
	\task are hardly a threat to the common culture.
	\task constitute the majority of the population.
\end{tasks}
\item Why are Arnold Schwarzenegger and Garth Brooks mentioned in Paragraph 5?
\begin{tasks}
	\task To prove their popularity around the world.
	\task To reveal the public's fear of immigrants.
	\task To give examples of successful immigrants.
	\task To show the powerful influence of American culture.
\end{tasks}
\item In the author's opinion, the absorption of immigrants into American society is
\begin{tasks}
	\task rewarding.
	\task successful.
	\task fruitless.
	\task harmful.
\end{tasks}