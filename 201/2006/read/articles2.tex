Stratford-on-Avon, as we all know, has only one industry – William Shakespeare – but there are two distinctly separate and increasingly hostile branches. There is the Royal Shakespeare Company (RSC), which presents superb productions of the plays at the Shakespeare Memorial Theatre on the Avon. And there are the townsfolk who largely live off the tourists who come, not to see the plays, but to look at Anne Hathaway's Cottage, Shakespeare's birthplace and the other sights.


The worthy residents of Stratford doubt that the theater adds a penny to their revenue. They frankly dislike the RSC's actors, them with their long hair and beards and sandals and noisiness. It's all deliciously ironic when you consider that Shakespeare, who earns their living, was himself an actor (with a beard) and did his share of noise-making.


The tourist streams are not entirely separate. The sightseers who come by bus – and often take in Warwick Castle and Blenheim Palace on the side – don't usually see the plays, and some of them are even surprised to find a theatre in Stratford. However, the playgoers do manage a little sight-seeing along with their playgoing. It is the playgoers, the RSC contends, who bring in much of the town's revenue because they spend the night (some of them four or five nights) pouring cash into the hotels and restaurants. The sightseers can take in everything and get out of town by nightfall.


The townsfolk don't see it this way and the local council does not contribute directly to the subsidy of the Royal Shakespeare Company. Stratford cries poor traditionally. Nevertheless every hotel in town seems to be adding a new wing or cocktail lounge. Hilton is building its own hotel there, which you may be sure will be decorated with Hamlet Hamburger Bars, the Lear Lounge, the Banquo Banqueting Room, and so forth, and will be very expensive.


Anyway, the townsfolk can't understand why the Royal Shakespeare Company needs a subsidy. (The theatre has broken attendance records for three years in a row. Last year its 1,431 seats were 94 per cent occupied all year long and this year they'll do better.) The reason, of course, is that costs have rocketed and ticket prices have stayed low.


It would be a shame to raise prices too much because it would drive away the young people who are Stratford's most attractive clientele. They come entirely for the plays, not the sights. They all seem to look alike (though they come from all over) – lean, pointed, dedicated faces, wearing jeans and sandals, eating their buns and bedding down for the night on the flagstones outside the theatre to buy the 20 seats and 80 standing-room tickets held for the sleepers and sold to them when the box office opens at 10:30 a.m.