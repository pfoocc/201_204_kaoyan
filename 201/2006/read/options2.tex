\item From the first two paragraphs, we learn that
\begin{tasks}
	\task the townsfolk deny the RSC's contribution to the town's revenue.
	\task the actors of the RSC imitate Shakespeare on and off stage.
	\task the two branches of the RSC are not on good terms.
	\task the townsfolk earn little from tourism.
\end{tasks}
\item It can be inferred from Paragraph 3 that
\begin{tasks}
	\task the sightseers cannot visit the Castle and the Palace separately.
	\task the playgoers spend more money than the sightseers.
	\task the sightseers do more shopping than the playgoers.
	\task the playgoers go to no other places in town than the theater.
\end{tasks}
\item By saying ``Stratford cries poor traditionally'' (Line 2, Paragraph 4), the author implies that
\begin{tasks}
	\task Stratford cannot afford the expansion projects.
	\task Stratford has long been in financial difficulties.
	\task the town is not really short of money.
	\task the townsfolk used to be poorly paid.
\end{tasks}
\item According to the townsfolk, the RSC deserves no subsidy because
\begin{tasks}
	\task ticket prices can be raised to cover the spending.
	\task the company is financially ill-managed.
	\task the behavior of the actors is not socially acceptable.
	\task the theatre attendance is on the rise.
\end{tasks}
\item From the text we can conclude that the author
\begin{tasks}
	\task is supportive of both sides.
	\task favors the townsfolk's view.
	\task takes a detached attitude.
	\task is sympathetic to the RSC.
\end{tasks}