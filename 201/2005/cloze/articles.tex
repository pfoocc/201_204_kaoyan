The human nose is an underrated tool. Humans are often thought to be insensitive  smellers  compared  with  animals, \uline{~~1~~} this  is  largely because, \uline{~~2~~} animals, we stand upright. This means that our noses are \uline{~~3~~} to perceiving those smells which float through the air, \uline{~~4~~} the majority of smells which stick to surfaces. In fact, \uline{~~5~~}, we are extremely sensitive to smells, \uline{~~6~~} we do not generally realize it. Our noses are capable of \uline{~~7~~} human smells even when these are \uline{~~8~~} to far below one part in one million.


Strangely, some people find that they can smell one type of flower but not another, \uline{~~9~~} others are sensitive to the smells of both flowers. This may be  because some people do not have the genes necessary to generate \uline{~~10~~} smell receptors in the nose. These receptors are the cells which sense smells and send \uline{~~11~~} to the brain. However, it has been found that even people insensitive to a certain smell \uline{~~12~~} can suddenly become sensitive to it when \uline{~~13~~} to it often enough.


The explanation for insensitivity to smell seems to be that the brain finds  it \uline{~~14~~} to keep all smell receptors working all the time but can \uline{~~15~~} new receptors if necessary. This may \uline{~~16~~} explain why we are not usually sensitive to our own smells – we simply do not need to be. We are not \uline{~~17~~} of the usual smell of our own house, but we \uline{~~18~~} new smells when we visit someone else's. The brain finds it  best to keep smell receptors \uline{~~19~~} for unfamiliar and emergency signals \uline{~~20~~} the smell of smoke, which might indicate the danger of fire.