\item According to Mc Whorter, the decline of formal English
\begin{tasks}
	\task is inevitable in radical education reforms.
	\task is but all too natural in language development.
	\task has caused the controversy over the counter-culture.
	\task brought about changes in public attitudes in the 1960s.
\end{tasks}
\item The word ``talking'' (Line 6, Paragraph 3) denotes
\begin{tasks}
	\task modesty.
	\task personality.
	\task liveliness.
	\task informality.
\end{tasks}
\item To which of the following statements would McWhorter most likely agree?
\begin{tasks}
	\task Logical thinking is not necessarily related to the way we talk.
	\task Black English can be more expressive than standard English.
	\task Non-standard varieties of human language are just as entertaining.
	\task Of all the varieties, standard English can best convey complex ideas.
\end{tasks}
\item The description of Russians' love of memorizing poetry shows the author's
\begin{tasks}
	\task interest in their language.
	\task appreciation of their efforts.
	\task admiration for their memory.
	\task contempt for their old-fashionedness.
\end{tasks}
\item According to the last paragraph, ``paper plates'' is to ``china'' as
\begin{tasks}
	\task ``temporary'' is to ``permanent''.
	\task ``radical'' is to ``conservative''.
	\task ``functional'' is to ``artistic''.
	\task ``humble'' is to ``noble''.
\end{tasks}