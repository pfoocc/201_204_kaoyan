\item In the opening paragraph, the author introduces his topic by
\begin{tasks}
	\task posing a contrast.
	\task justifying an assumption.
	\task making a comparison.
	\task explaining a phenomenon.
\end{tasks}
\item The statement ``it is all too monkey'' (Last line, Paragraph l) implies that
\begin{tasks}
	\task monkeys are also outraged by slack rivals.
	\task resenting unfairness is also monkeys' nature.
	\task monkeys, like humans, tend to be jealous of each other.
	\task no animals other than monkeys can develop such emotions.
\end{tasks}
\item Female capuchin monkeys were chosen for the research most probably because they are
\begin{tasks}
	\task more inclined to weigh what they get.
	\task attentive to researchers' instructions.
	\task nice in both appearance and temperament.
	\task more generous than their male companions.
\end{tasks}
\item Dr. Brosnan and Dr. de Waal have eventually found in their study that the monkeys
\begin{tasks}
	\task prefer grapes to cucumbers.
	\task can be taught to exchange things.
	\task will not be co-operative if feeling cheated.
	\task are unhappy when separated from others.
\end{tasks}
\item What can we infer from the last paragraph?
\begin{tasks}
	\task Monkeys can be trained to develop social emotions.
	\task Human indignation evolved from an uncertain source.
	\task Animals usually show their feelings openly as humans do.
	\task Cooperation among monkeys remains stable only in the wild.
\end{tasks}