Everybody loves a fat pay rise. Yet pleasure at your own can vanish if you learn that a colleague has been given a bigger one. Indeed, if he has a reputation for slacking, you might even be outraged. Such behaviour is regarded as ``all too human'', with the underlying assumption that other animals would not be capable of this finely developed sense of grievance. But a study by Sarah Brosnan and Frans de Waal of Emory University in Atlanta, Georgia, which has just been published in Nature, suggests that it is all too monkey, as well.


The researchers studied the behaviour of female brown capuchin monkeys. They look cute. They are good-natured, co-operative creatures, and they share their food readily. Above all, like their female human counterparts, they tend to pay much closer attention to the value of ``goods and services'' than males.


Such characteristics make them perfect candidates for Dr. Brosnan's and Dr. de Waal's study. The researchers spent two years teaching their monkeys to exchange tokens for food. Normally, the monkeys were happy enough to exchange pieces of rock for slices of cucumber. However, when two monkeys were placed in separate but adjoining chambers, so that each could observe what the other was getting in return for its rock, their behaviour became markedly different.


In the world of capuchins grapes are luxury goods (and much preferable to cucumbers). So when one monkey was handed a grape in exchange for her token, the second was reluctant to hand hers over for a mere piece of cucumber. And if one received a grape without having to provide her token in exchange at all, the other either tossed her own token at the researcher or out of the chamber, or refused to accept the slice of cucumber. Indeed, the mere presence of a grape in the other chamber (without an actual monkey to eat it) was enough to induce resentment in a female capuchin.


The researchers suggest that capuchin monkeys, like humans, are guided by social emotions. In the wild, they are a co-operative, group-living species. Such co-operation is likely to be stable only when each animal feels it is not being cheated. Feelings of righteous indignation, it seems, are not the preserve of people alone. Refusing a lesser reward completely makes these feelings abundantly clear to other members of the group. However, whether such a sense of fairness evolved independently in capuchins and humans, or whether it stems from the common ancestor that the species had 35 million years ago, is, as yet, an unanswered question.