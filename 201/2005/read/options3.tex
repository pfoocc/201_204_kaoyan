\item Researchers have come to believe that dreams
\begin{tasks}
	\task can be modified in their courses.
	\task are susceptible to emotional changes.
	\task reflect our innermost desires and fears.
	\task are a random outcome of neural repairs.
\end{tasks}
\item By referring to the limbic system, the author intends to show
\begin{tasks}
	\task its function in our dreams.
	\task the mechanism of REM sleep.
	\task the relation of dreams to emotions.
	\task its difference from the prefrontal cortex.
\end{tasks}
\item The negative feelings generated during the day tend to
\begin{tasks}
	\task aggravate in our unconscious mind.
	\task develop into happy dreams.
	\task persist till the time we fall asleep.
	\task show up in dreams early at night.
\end{tasks}
\item Cartwright seems to suggest that
\begin{tasks}
	\task waking up in time is essential to the ridding of bad dreams.
	\task visualizing bad dreams helps bring them under control.
	\task dreams should be left to their natural progression.
	\task dreaming may not entirely belong to the unconscious.
\end{tasks}
\item What advice might Cartwright give to those who sometimes have bad dreams?
\begin{tasks}
	\task Lead your life as usual.
	\task Seek professional help.
	\task Exercise conscious control.
	\task Avoid anxiety in the daytime.
\end{tasks}