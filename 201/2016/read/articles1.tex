France, which prides itself as the global innovator of fashion, has decided its fashion industry has lost an absolute right to define physical beauty for women. Its lawmakers gave preliminary approval last week to a law that would make it a crime to employ ultra-thin models on runways. The parliament also agreed to ban websites that ``incite excessive thinness'' by promoting extreme dieting.


Such measures have a couple of uplifting motives. They suggest beauty should not be defined by looks that end up impinging on health. That's a start. And the ban on ultra-thin models seems to go beyond protecting models from starving themselves to death – as some have done. It tells the fashion industry that it must take responsibility for the signal it sends women, especially teenage girls, about the social tape-measure they must use to determine their individual worth.


The bans, if fully enforced, would suggest to women (and many men) that they should not let others be arbiters of their beauty. And perhaps faintly, they hint that people should look to intangible qualities like character and intellect rather than dieting their way to size zero or wasp-waist physiques.


The French measures, however, rely too much on severe punishment to change a culture that still regards beauty as skin-deep – and bone-showing. Under  the law, using a fashion model that does not meet a government-defined index of body mass could result in a \$85,000 fine and six months in prison.


The fashion industry knows it has an inherent problem in focusing on material adornment and idealized body types. In Denmark, the United States, and a few other countries, it is trying to set voluntary standards for models and fashion images that rely more on peer pressure for enforcement.


In contrast to France's actions, Denmark's fashion industry agreed last month on rules and sanctions regarding the age, health, and other characteristics of models. The newly revised Danish Fashion Ethical Charter clearly states: ``We are aware of and take responsibility for the impact the fashion industry has on body ideals, especially on young people.'' The charter's main tool of enforcement is to deny access for designers and modeling agencies to Copenhagen Fashion Week (CFW), which is run by the Danish Fashion Institute. But in general it relies on a name-and-shame method of compliance.


Relying on ethical persuasion rather than law to address the misuse of body ideals may be the best step. Even better would be to help elevate notions of beauty beyond the material standards of a particular industry.