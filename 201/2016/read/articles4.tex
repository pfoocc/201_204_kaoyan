There will eventually come a day when The New York Times ceases to publish stories on newsprint. Exactly when that day will be is a matter of debate. ``Sometime in the future,'' the paper's publisher said back in 2010.


Nostalgia for ink on paper and the rustle of pages aside, there's plenty of incentive to ditch print. The infrastructure required to make a physical newspaper– printing presses, delivery trucks – isn't just expensive; it's excessive at a time when online-only competitors don't have the same set of financial constraints. Readers are migrating away from print anyway. And though print ad sales still dwarf their online and mobile counterparts, revenue from print is still declining.


Overhead may be high and circulation lower, but rushing to eliminate its print edition would be a mistake, says BuzzFeed CEO Jonah Peretti.


Peretti says the Times shouldn't waste time getting out of the print business, but only if they go about doing it the right way. ``Figuring out a way to accelerate that transition would make sense for them,'' he said, ``but if you discontinue it, you're going to have your most loyal customers really upset with you.''


Sometimes that's worth making a change anyway. Peretti gives the example of Netflix discontinuing its DVD-mailing service to focus on streaming. ``It was seen as a blunder,'' he said. The move turned out to be foresighted. And if Peretti were in charge at the Times? ``I wouldn't pick a year to end print,'' he said. ``I would raise prices and make it into more of a legacy product.''


The most loyal customers would still get the product they favor, the idea goes, and they'd feel like they were helping sustain the quality of something they believe in. ``So if you're overpaying for print, you could feel like you were helping,'' Peretti said. ``Then increase it at a higher rate each year and essentially try to generate additional revenue.'' In other words, if you're going to make a print product, make it for the people who are already obsessed with it. Which may be what the Times is doing already. Getting the print edition seven days a week costs nearly \$500 a year – more than twice as much as a digital-only subscription.


``It's a really hard thing to do and it's a tremendous luxury that BuzzFeed doesn't have a legacy business,'' Peretti remarked. ``But we're going to have questions like that where we have things we're doing that don't make sense when  the market changes and the world changes. In those situations, it's better to be more aggressive than less aggressive.''