\item The New York Times is considering ending its print edition partly due to
\begin{tasks}
	\task the high cost of operation.
	\task the increasing online ad sales.
	\task the pressure from its investors.
	\task the complaints from its readers.
\end{tasks}
\item Peretti suggests that, in face of the present situation, the Times should
\begin{tasks}
	\task end the print edition for good.
	\task make strategic adjustments.
	\task seek new sources of readership.
	\task aim for efficient management.
\end{tasks}
\item It can be inferred from Paragraphs 5 and 6 that a ``legacy product''
\begin{tasks}
	\task helps restore the glory of former times.
	\task is meant for the most loyal customers.
	\task will have the cost of printing reduced.
	\task expands the popularity of the paper.
\end{tasks}
\item Peretti believes that, in a changing world,
\begin{tasks}
	\task traditional luxuries can stay unaffected.
	\task cautiousness facilitates problem-solving.
	\task aggressiveness better meets challenges.
	\task legacy businesses are becoming outdated.
\end{tasks}
\item Which of the following would be the best title of the text?
\begin{tasks}
	\task Shift to Online Newspapers All at Once
	\task Make Your Print Newspaper a Luxury Good
	\task Keep Your Newspapers Forever in Fashion
	\task Cherish the Newspaper Still in Your Hand
\end{tasks}
