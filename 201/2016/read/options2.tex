\item Britain's public sentiment about the countryside
\begin{tasks}
	\task is not well reflected in politics.
	\task is fully backed by the royal family.
	\task didn't start till the Shakespearean age.
	\task has brought much benefit to the NHS.
\end{tasks}
\item According to Paragraph 2, the achievements of the National Trust are now being
\begin{tasks}
	\task largely overshadowed.
	\task properly protected.
	\task effectively reinforced.
	\task gradually destroyed.
\end{tasks}
\item Which of the following can be inferred from Paragraph 3?
\begin{tasks}
	\task Labour is under attack for opposing development.
	\task The Conservatives may abandon ``off-plan'' building.
	\task Ukip may gain from its support for rural conservation.
	\task The Liberal Democrats are losing political influence.
\end{tasks}
\item The author holds that George Osborne's preference
\begin{tasks}
	\task shows his disregard for the character of rural areas.
	\task stresses the necessity of easing the housing crisis.
	\task highlights his firm stand against lobby pressure.
	\task reveals a strong prejudice against urban areas.
\end{tasks}
\item In the last paragraph, the author shows his appreciation of
\begin{tasks}
	\task the size of population in Britain.
	\task the enviable urban lifestyle in Britain.
	\task the town-and-country planning in Britain.
	\task the political life in today's Britain.
\end{tasks}