\item The author views Milton Friedman's statement about CSR with
\begin{tasks}
	\task tolerance.
	\task skepticism.
	\task approval.
	\task uncertainty.
\end{tasks}
\item According to Paragraph 2, CSR helps a company by
\begin{tasks}
	\task guarding it against malpractices.
	\task protecting it from being defamed.
	\task winning trust from consumers.
	\task raising the quality of its products.
\end{tasks}
\item The expression ``more lenient'' (Para.4) is closest in meaning to
\begin{tasks}
	\task less controversial.
	\task more effective.
	\task more lasting.
	\task less severe.
\end{tasks}
\item When prosecutors evaluate a case, a company's CSR record
\begin{tasks}
	\task has an impact on their decision.
	\task comes across as reliable evidence.
	\task increases the chance of being penalized.
	\task constitutes part of the investigation.
\end{tasks}
\item Which of the following is true of CSR, according to the last paragraph?
\begin{tasks}
	\task Its negative effects on businesses are often overlooked.
	\task The necessary amount of companies' spending on it is unknown.
	\task Companies' financial capacity for it has been overestimated.
	\task It has brought much benefit to the banking industry.
\end{tasks}