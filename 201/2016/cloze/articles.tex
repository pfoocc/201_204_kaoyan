In Cambodia, the choice of a spouse is a complex one for the young male. It may involve not only his parents and his friends, \uline{~~1~~} those of the young woman,   but also a matchmaker. A young man can \uline{~~2~~} a likely spouse on his own and then ask his parents to \uline{~~3~~} the marriage negotiations, or the young man's   parents may make the choice of a spouse, giving the child little to say in the selection. \uline{~~4~~}, a girl may veto the spouse her parents have chosen. \uline{~~5~~} a spouse has been selected, each family investigates the other to make sure its child is marrying \uline{~~6~~} a good family.


The traditional wedding is a long and colorful affair. Formerly it lasted three days, \uline{~~7~~} by the 1980s it more commonly lasted a day and a half. Buddhist priests offer a short sermon and \uline{~~8~~} prayers of blessing. Parts of the ceremony involve ritual hair cutting, \uline{~~9~~} cotton threads soaked in holy water around the bride's and groom's wrists, and \uline{~~10~~} a candle around a circle of happily married and respected couples to bless the \uline{~~11~~}. Newlyweds traditionally move in with the wife's parents and may \uline{~~12~~} with them up to a year, \uline{~~13~~} they can build a new house nearby.


Divorce is legal and easy to \uline{~~14~~}, but not common. Divorced persons are \uline{~~15~~} with some disapproval. Each spouse retains \uline{~~16~~} property he or she G \uline{~~17~~} into the marriage, and jointly-acquired property is \uline{~~18~~} equally. Divorced persons may remarry, but a gender prejudice \uline{~~19~~} up: The divorced male doesn't have a waiting period before he can remarry \uline{~~20~~} the woman must wait ten months.