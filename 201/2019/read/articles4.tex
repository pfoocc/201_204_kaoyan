States will be able to force more people to pay sales tax when they make online purchases under a Supreme Court decision Thursday that will leave shoppers with lighter wallets but is a big financial win for states.


The Supreme Court's opinion Thursday overruled a pair of decades-old decisions that states said cost them billions of dollars in lost revenue annually. The decisions made it more difficult for states to collect sales tax on certain online purchases.


The cases the court overturned said that if a business was shipping a customer's purchase to a state where the business didn't have a physical presence such as a warehouse or office, the business didn't have to collect sales tax for the state. Customers were generally responsible for paying the sales tax to the state themselves if they weren't charged it, but most didn't realize they owed it and few paid.


Justice Anthony Kennedy wrote that the previous decisions were flawed. ``Each year the physical presence rule becomes further removed from economic reality and results in significant revenue losses to the States,'' he wrote in an opinion joined by four other justices. Kennedy wrote that the rule ``limited states' ability to seek long-term prosperity and has prevented market participants from competing on an even playing field.''


The ruling is a victory for big chains with a presence in many states, since they usually collect sales tax on online purchases already. Now, rivals will be charging sales tax where they hadn't before. Big chains have been collecting sales tax nationwide because they typically have physical stores in whatever state a purchase is being shipped to. Amazon.com, with its network of warehouses, also collects sales tax in every state that charges it, though third-party sellers who use the site don't have  to.


Until now, many sellers that have a physical presence in only a single state or a few states have been able to avoid charging sales taxes when they ship to addresses outside those states. Sellers that use eBay and Etsy, which provide platforms for smaller sellers, also haven't been collecting sales tax nationwide. Under the ruling Thursday, states can pass laws requiring out-of-state sellers to collect the state's sales tax from customers and send it to the state.


Retail trade groups praised the ruling, saying it levels the playing field for local and online businesses. The losers, said retail analyst Neil Saunders, are online-only retailers, especially smaller ones. Those retailers may face headaches complying with various state sales tax laws. The Small Business \& Entrepreneurship Council advocacy group said in a statement, ``Small businesses and internet entrepreneurs are not well served at all by this decision.''