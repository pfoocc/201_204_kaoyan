Grade inflation – the gradual increase in average GPAs (grade-point averages) over the past few decades – is often considered a product of a consumer era in higher education, in which students are treated like customers to be pleased. But another, related force – a policy often buried deep in course catalogs called ``grade forgiveness'' – is helping raise GPAs.


Grade forgiveness allows students to retake a course in which they received a low grade, and the most recent grade or the highest grade is the only one that counts in calculating a student's overall GPA.


The use of this little-known practice has accelerated in recent years, as colleges continue to do their utmost to keep students in school (and paying tuition) and improve their graduation rates. When this practice first started decades ago, it was usually limited to freshmen, to give them a second chance to take a class in their first year if they struggled in their transition to college-level courses. But now most colleges, save for many selective campuses, allow all undergraduates, and even graduate students, to get their low grades forgiven.


College officials tend to emphasize that the goal of grade forgiveness is less about the grade itself and more about encouraging students to retake courses critical to their degree program and graduation without incurring a big penalty. ``Ultimately,'' said Jack Miner, Ohio State University's registrar, ``we see students achieve more success because they retake a course and do better in subsequent courses or master the content that allows them to graduate on time.''


That said, there is a way in which grade forgiveness satisfies colleges' own needs as well. For public institutions, state funds are sometimes tied partly to their success on metrics such as graduation rates and student retention – so better grades can, by boosting figures like those, mean more money. And anything that raises GPAs will likely make students – who, at the end of the day, are paying the bill – feel they've gotten a better value for their tuition dollars, which is another big concern for colleges.


Indeed, grade forgiveness is just another way that universities are responding to consumers' expectations for higher education. Since students and parents expect a college degree to lead to a job, it is in the best interest of a school to turn out graduates who are as qualified as possible – or at least appear to be. On this, students' and colleges' incentives seem to be aligned.