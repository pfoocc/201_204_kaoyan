\item According to Paragraph 1, one motive in imposing the new rule is to
\begin{tasks}
	\task enhance bankers' sense of responsibility.
	\task help corporations achieve larger profits.
	\task build a new system of financial regulation.
	\task guarantee the bonuses of top executives.
\end{tasks}
\item Alfred Marshall is quoted to indicate
\begin{tasks}
	\task the conditions for generating quick profits.
	\task governments' impatience in decision-making.
	\task the solid structure of publicly traded companies.
	\task ``short-termism'' in economic activities.
\end{tasks}
\item It is argued that the influence of transient investment on public companies can be
\begin{tasks}
	\task indirect.
	\task adverse.
	\task minimal.
	\task temporary.
\end{tasks}
\item The US and France examples are used to illustrate
\begin{tasks}
	\task the obstacles to preventing ``short-termism''.
	\task the significance of long-term thinking.
	\task the approaches to promoting ``long-termism''.
	\task the prevalence of short-term thinking.
\end{tasks}
\item Which of the following would be the best title for the text?
\begin{tasks}
	\task Failure of Quarterly Capitalism
	\task Patience as a Corporate Virtue
	\task Decisiveness Required of Top Executives
	\task Frustration of Risk-taking Bankers
\end{tasks}