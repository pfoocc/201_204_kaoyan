In his book The Tipping Point, Malcolm Gladwell argues that ``social epidemics'' are driven in large part by the actions of a tiny minority of special individuals, often called influentials, who are unusually informed, persuasive, or well connected. The idea is intuitively compelling, but it doesn't explain how ideas actually spread.


The supposed importance of influentials derives from a plausible-sounding but largely untested theory called the ``two-step flow of communication'': Information flows from the media to the influentials and from them to everyone else. Marketers have embraced the two-step flow because it suggests that if they can just find and influence the influentials, those select people will do most of the work for them. The theory also seems to explain the sudden and unexpected popularity of certain looks, brands, or neighborhoods. In many such cases, a cursory search for causes finds that some small group of people was wearing, promoting, or developing whatever it is before anyone else paid attention. Anecdotal evidence of this kind fits nicely with the idea that only certain special people can drive trends.


In their recent work, however, some researchers have come up with the finding that influentials have far less impact on social epidemics than is generally supposed. In fact, they don't seem to be required at all.


The researchers' argument stems from a simple observation about social influence: With the exception of a few celebrities like Oprah Winfrey – whose outsize presence is primarily a function of media, not interpersonal, influence – even the most influential members of a population simply don't interact with that many others. Yet it is precisely these non-celebrity influentials who, according to the two-step-flow theory, are supposed to drive social epidemics, by influencing their friends and colleagues directly. For a social epidemic to occur, however, each person so affected must then influence his or her own acquaintances, who must in turn influence theirs, and so on; and just how many others pay attention to each of these people has little to do with the initial influential. If people in the network just two degrees removed from the initial influential prove resistant, for example, the cascade of change won't propagate very far or affect many people.


Building on this basic truth about interpersonal influence, the researchers studied the dynamics of social influence by conducting thousands of computer simulations of populations, manipulating a number of variables relating to people's ability to influence others and their tendency to be influenced. They found that the principal requirement for what is called ``global cascades'' – the widespread propagation of influence through networks – is the presence not of a few influentials but, rather, of a critical mass of easily influenced people.