\item Bankers complained that they were forced to
\begin{tasks}
	\task follow unfavorable asset evaluation rules.
	\task collect payments from third parties.
	\task cooperate with the price managers.
	\task reevaluate some of their assets.
\end{tasks}
\item According to the author, the rule changes of the FASB may result in
\begin{tasks}
	\task the diminishing role of management.
	\task the revival of the banking system.
	\task the banks' long-term asset losses.
	\task the weakening of its independence.
\end{tasks}
\item According to Paragraph 4, McCreevy objects to the IASB's attempt to
\begin{tasks}
	\task keep away from political influences.
	\task evade the pressure from their peers.
	\task act on their own in rule-setting.
	\task take gradual measures in reform.
\end{tasks}
\item The author thinks the banks were ``on the wrong planet'' in that they
\begin{tasks}
	\task misinterpreted market price indicators.
	\task exaggerated the real value of their assets.
	\task neglected the likely existence of bad debts.
	\task denied booking losses in their sale of assets.
\end{tasks}
\item The author's attitude towards standard-setters is one of
\begin{tasks}
	\task satisfaction.
	\task skepticism.
	\task objectiveness.
	\task sympathy.
\end{tasks}