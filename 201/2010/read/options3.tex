\item By citing the book The Tipping Point, the author intends to
\begin{tasks}
	\task analyze the consequences of social epidemics.
	\task discuss influentials' function in spreading ideas.
	\task exemplify people's intuitive response to social epidemics.
	\task describe the essential characteristics of influentials.
\end{tasks}
\item The author suggests that the ``two-step-flow theory''
\begin{tasks}
	\task serves as a solution to marketing problems.
	\task has helped explain certain prevalent trends.
	\task has won support from influentials.
	\task requires solid evidence for its validity.
\end{tasks}
\item What the researchers have observed recently shows that
\begin{tasks}
	\task the power of influence goes with social interactions.
	\task interpersonal links can be enhanced through the media.
	\task influentials have more channels to reach the public.
	\task most celebrities enjoy wide media attention.
\end{tasks}
\item The underlined phrase ``these people'' in Paragraph 4 refers to the ones who
\begin{tasks}
	\task stay outside the network of social influence.
	\task have little contact with the source of influence.
	\task are influenced and then influence others.
	\task are influenced by the initial influential.
\end{tasks}
\item What is the essential element in the dynamics of social influence?
\begin{tasks}
	\task The eagerness to be accepted.
	\task The impulse to influence others.
	\task The readiness to be influenced.
	\task The inclination to rely on others.
\end{tasks}