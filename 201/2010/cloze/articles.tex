In 1924 America's National Research Council sent two engineers to supervise a series of experiments at a telephone-parts factory called the Hawthorne Plant near Chicago. It hoped they would learn how shop-floor lighting \uline{~~1~~} workers' productivity. Instead, the studies ended \uline{~~2~~} giving their name to the ``Hawthorne effect,'' the extremely influential idea that the very \uline{~~3~~} of being experimented  upon changed subjects' behavior.


The idea arose because of the \uline{~~4~~} behavior of the women in the plant. According to \uline{~~5~~} of the experiments, their hourly output rose when lighting was increased, but also when it was dimmed. It did not \uline{~~6~~} what was done in  the experiment; \uline{~~7~~} something was changed, productivity rose. A(n) \uline{~~8~~} that they were being experimented upon seemed to be \uline{~~9~~} to alter workers' behavior \uline{~~10~~} itself.


After several decades, the same data were \uline{~~11~~} to econometric analysis. The Hawthorne experiments had another surprise in store. \uline{~~12~~} the descriptions on  record, no systematic \uline{~~13~~} was found that levels of productivity were related to changes in lighting.


It turns out that the peculiar way of conducting the experiments may have led to \uline{~~14~~} interpretations of what happened. \uline{~~15~~}, lighting was always changed on a Sunday. When work started again on Monday, output \uline{~~16~~} rose compared with the previous Saturday and \uline{~~17~~} to rise for the next couple of days. \uline{~~18~~}, a comparison with data for weeks when there was no experimentation showed that output always went up on Mondays. Workers \uline{~~19~~} to be diligent for the first few days of the week in any case, before \uline{~~20~~} a plateau and then slackening off. This suggests that the alleged ``Hawthorne effect'' is hard to pin down.