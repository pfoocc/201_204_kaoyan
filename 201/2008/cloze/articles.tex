The idea that some groups of people may be more intelligent than others is one of those hypotheses that dare not speak its name. But Gregory Cochran is \uline{~~1~~} to say it anyway. He is that \uline{~~2~~} bird, a scientist who works independently \uline{~~3~~} any institution. He helped popularize the idea that some diseases not \uline{~~4~~} thought to have a bacterial cause were actually infections, which aroused much controversy when it was first suggested.


\uline{~~5~~} he, however, might tremble at the \uline{~~6~~} of what he is about to do. Together with another two scientists, he is publishing a paper which not only \uline{~~7~~} that one  group of humanity is more intelligent than the others, but explains the process that has brought this about. The group in \uline{~~8~~} are a particular people originated from  central Europe. The process is natural selection.


This group generally do well in IQ test, \uline{~~9~~} 12-15 points above the \uline{~~10~~} value of 100, and have contributed \uline{~~11~~} to the intellectual and cultural life of the  We s t, as th e \uline{~~12~~} of th e ir eli t es, in c lu d i n g sev e r al wo r ld - r e no w n ed scientists, \uline{~~13~~}. They also suffer more often than most people from a number of nasty genetic diseases, such as breast cancer. These facts, \uline{~~14~~}, have previously been thought unrelated. The former has been \uline{~~15~~} to social effects,   such as a strong tradition of \uline{~~16~~} education. The latter was seen as a (an) \uline{~~17~~} of genetic isolation. Dr. Cochran suggests that the intelligence and diseases are intimately \uline{~~18~~}.  His argument is that the unusual history of  these people  has \uline{~~19~~} them  to  unique  evolutionary  pressures  that  have  resulted  in this \uline{~~20~~} state of affairs.