\item In the first paragraph, the author discusses
\begin{tasks}
	\task the background information of journal editing.
	\task the publication routine of laboratory reports.
	\task the relations of authors with journal publishers.
	\task the traditional process of journal publication.
\end{tasks}
\item Which of the following is true of the OECD report?
\begin{tasks}
	\task It criticizes government-funded research.
	\task It introduces an effective means of publication.
	\task It upsets profit-making journal publishers.
	\task It benefits scientific research considerably.
\end{tasks}
\item According to the text, online publication is significant in that
\begin{tasks}
	\task it provides an easier access to scientific results.
	\task it brings huge profits to scientific researchers.
	\task it emphasizes the crucial role of scientific knowledge.
	\task it facilitates public investment in scientific research.
\end{tasks}
\item With the open-access publishing model, the author of a paper is required to
\begin{tasks}
	\task cover the cost of its publication.
	\task subscribe to the journal publishing it.
	\task allow other online journals to use it freely.
	\task complete the peer-review before submission.
\end{tasks}
\item Which of the following best summarizes the text?
\begin{tasks}
	\task The Internet is posing a threat to publishers.
	\task A new mode of publication is emerging.
	\task Authors welcome the new channel for publication.
	\task Publication is rendered easier by online service.
\end{tasks}