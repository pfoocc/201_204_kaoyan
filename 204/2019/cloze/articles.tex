Weighing yourself regularly is a wonderful way to stay aware of any significant weight fluctuations. \uline{~~1~~} , when done too often, this habit can sometimes hurt more than it \uline{~~2~~} .


As for me, weighing myself every day caused me to shift my focus from being generally healthy and physically active, to focusing \uline{~~3~~} on the scale. That was bad to my overall fitness goals. I had gained weight in the form of muscle mass, but thinking only of \uline{~~4~~} the number on the scale, I altered my training program. That conflicted with how I needed to train to \uline{~~5~~} my goals.


I also found that weighing myself daily did not provide an accurate \uline{~~6~~} of the hard work and progress I was making in the gym. It takes about three weeks to a month to notice significant changes in your weight \uline{~~7~~} altering your training program. The most \uline{~~8~~} changes will be observed in skill level, strength and inches lost.


For these \uline{~~9~~} , I stopped weighing myself every day and switched to a bimonthly weighing schedule \uline{~~10~~} . Since weight loss is not my goal, it is less important for me to \uline{~~11~~} my weight each week. Weighing every other week allows me to observe and \uline{~~12~~} any significant weight changes. That tells me whether I need to \uline{~~13~~} my training program.


I use my bimonthly weigh-in \uline{~~14~~} to get information about my nutrition as well. If my training intensity remains the same, but I'm constantly \uline{~~15~~} and dropping weight, this is a \uline{~~16~~} that I need to increase my daily caloric intake.


The \uline{~~17~~} to stop weighing myself every day has done wonders for my overall health, fitness and well-being. I'm experiencing increased zeal for working out since I no longer carry the burden of a \uline{~~18~~} morning weigh-in. I've also experienced greater success in achieving my specific fitness goals, \uline{~~19~~} I'm training according to those goals, not the numbers on a scale.


Rather than \uline{~~20~~} over the scale, turn your focus to how you look, feel, how your clothes fit and your overall energy level.