\item Researchers think that guilt can be a good thing because it may help \uline{~~~~}.
\begin{tasks}
	\task regulate a child's basic emotions
	\task improve a child's intellectual ability
	\task foster a child's moral development
	\task intensify a child's positive feelings
\end{tasks}
\item According to Paragraph 2, many people still consider guilt to be \uline{~~~~}.
\begin{tasks}
	\task deceptive
	\task burdensome
	\task addictive
	\task inexcusable
\end{tasks}
\item Vaish holds that the rethinking about guilt comes from an awareness that \uline{~~~~}.
\begin{tasks}
	\task emotions are context-independent
	\task emotions are socially constructive
	\task emotional stability can benefit health
	\task an emotion can play opposing roles
\end{tasks}
\item Malti and others have shown that cooperation and sharing \uline{~~~~}.
\begin{tasks}
	\task may help correct emotional deficiencies
	\task can result from either sympathy or guilt
	\task can bring about emotional satisfaction
	\task may be the outcome of impulsive acts
\end{tasks}
\item The word ``transgressions'' (Line 4, Para.5) is closest in meaning to \uline{~~~~}.
\begin{tasks}
	\task teachings
	\task discussions
	\task restrictions
	\task wrongdoings
\end{tasks}