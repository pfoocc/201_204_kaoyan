Happy people work differently. They're more productive, more creative, and willing to take greater risks. And new research suggests that happiness might influence \uline{~~1~~} firms work, too.


Companies located in places with happier people invest more, according to a recent research paper. \uline{~~2~~} , firms in happy places spend more on R\&D (research and development). That's because happiness is linked to the kind of longer-term thinking \uline{~~3~~} for making investments for the future.


The researchers wanted to know if the \uline{~~4~~} and inclination for risk-taking that come with happiness would \uline{~~5~~} the way companies invested. So they compared U.S. cities' average happiness \uline{~~6~~} by Gallup polling with the investment activity of publicly traded firms in those areas.


\uline{~~7~~} enough, firms' investment and R\&D intensity were correlated with the happiness of the area in which they were \uline{~~8~~} . But is it really happiness that's linked to investment, or could something else about happier cities \uline{~~9~~} why firms there spend more on R\&D? To find out, the researchers controlled for various \uline{~~10~~} that might make firms more likely to invest – like size, industry, and sales – and for indicators that a place was \uline{~~11~~} to live in, like growth in wages or population. The link between happiness and investment generally \uline{~~12~~} even after accounting for these things.


The correlation between happiness and investment was particularly strong for younger firms, which the authors \uline{~~13~~} to ``less codified decision making process'' and the possible presence of ``younger and less \uline{~~14~~} managers who are more likely to be influenced by sentiment.'' The relationship was \uline{~~15~~} stronger in places where happiness was spread more \uline{~~16~~} . Firms seem to invest more in places where most people are relatively happy, rather than in places with happiness inequality.


\uline{~~17~~} this doesn't prove that happiness causes firms to invest more or to take a longer-term view, the authors believe it at least \uline{~~18~~} at that possibility. It's not hard to imagine that local culture and sentiment would help \uline{~~19~~} how executives think about the future. ``It surely seems plausible that happy people would be more forward-thinking and creative and \uline{~~20~~} R\&D more than the average,'' said one researcher.