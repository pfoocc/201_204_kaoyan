\item The usual time-management techniques don't work because \uline{~~~~}.
\begin{tasks}
	\task what they can offer does not ease the modern mind
	\task what challenging books demand is repetitive reading
	\task what people often forget is carrying a book with them
	\task what deep reading requires cannot be guaranteed
\end{tasks}
\item The ``empty bottles'' metaphor illustrates that people feel a pressure to \uline{~~~~}.
\begin{tasks}
	\task update their to-do lists
	\task make passing time fulfilling
	\task carry their plans through
	\task pursue carefree reading
\end{tasks}
\item Eberle would agree that scheduling regular times for reading helps \uline{~~~~}.
\begin{tasks}
	\task encourage the efficiency mind-set
	\task develop online reading habits
	\task promote ritualistic reading
	\task achieve immersive reading
\end{tasks}
\item ``Carry a book with you at all times'' can work if \uline{~~~~}.
\begin{tasks}
	\task reading becomes your primary business of the day
	\task all the daily business has been promptly dealt with
	\task you are able to drop back to business after reading
	\task time can be evenly split for reading and business
\end{tasks}
\item The best title for this text could be \uline{~~~~}.
\begin{tasks}
	\task How to Enjoy Easy Reading
	\task How to Find Time to Read
	\task How to Set Reading Goals
	\task How to Read Extensively
\end{tasks}