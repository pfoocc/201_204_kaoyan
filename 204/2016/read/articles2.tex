Biologists estimate that as many as 2 million lesser prairie chickens – a kind of bird living on stretching grasslands – once lent red to the often grey landscape  of the midwestern and southwestern United States. But just some 22,000 birds remain today, occupying about 16% of the species' historic range.


The crash was a major reason the U.S. Fish and Wildlife Service (USFWS) decided to formally list the bird as threatened. ``The lesser prairie chicken is in a desperate situation,'' said USFWS Director Daniel Ashe. Some environmentalists, however, were disappointed. They had pushed the agency to designate the bird as ``endangered,'' a status that gives federal officials greater regulatory power to crack down on threats. But Ashe and others argued that the ``threatened'' tag gave the federal government flexibility to try out new, potentially less confrontational conservation approaches. In particular, they called for forging closer collaborations with western state governments, which are often uneasy with federal action, and with the private landowners who control an estimated 95% of the prairie chicken's habitat.


Under the plan, for example, the agency said it would not prosecute landowners or businesses that unintentionally kill, harm, or disturb the bird, as long as they had signed a range-wide management plan to restore prairie chicken habitat. Negotiated by USFWS and the states, the plan requires individuals and businesses that damage habitat as part of their operations to pay into a fund to replace every acre destroyed with 2 new acres of suitable habitat. The fund will  also be used to compensate landowners who set aside habitat. USFWS also set an interim goal of restoring prairie chicken populations to an annual average of 67,000 birds over the next 10 years. And it gives the Western Association of Fish and Wildlife Agencies (WAFWA), a coalition of state agencies, the job of monitoring progress. Overall, the idea is to let ``states remain in the driver's seat for managing the species,'' Ashe said.


Not everyone buys the win-win rhetoric. Some Congress members are trying to block the plan, and at least a dozen industry groups, four states, and three environmental groups are challenging it in federal court. Not surprisingly, industry groups and states generally argue it goes too far; environmentalists say it doesn't go far enough. ``The federal government is giving responsibility for managing the bird to the same industries that are pushing it to extinction,'' says biologist Jay Lininger.