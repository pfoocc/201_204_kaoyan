\item One cross-generation mark of a successful life is \uline{~~~~}.
\begin{tasks}
	\task trying out different lifestyles
	\task having a family with children
	\task working beyond retirement age
	\task setting up a profitable business
\end{tasks}
\item It can be learned from Paragraph 3 that young people tend to \uline{~~~~}.
\begin{tasks}
	\task favor a slower life pace
	\task hold an occupation longer
	\task attach importance to pre-marital finance
	\task give priority to childcare outside the home
\end{tasks}
\item The priorities and expectations defined by the young will \uline{~~~~}.
\begin{tasks}
	\task become increasingly clear
	\task focus on materialistic issues
	\task depend largely on political preferences
	\task reach almost all aspects of American life
\end{tasks}
\item Both young and old agree that \uline{~~~~}.
\begin{tasks}
	\task good-paying jobs are less available
	\task the old made more life achievements
	\task housing loans today are easy to obtain
	\task getting established is harder for the young
\end{tasks}
\item Which of the following is true about Schneider?
\begin{tasks}
	\task He found a dream job after graduating from college.
	\task His parents believe working steadily is a must for success.
	\task His parents' good life has little to do with a college degree.
	\task He thinks his job as a technician quite challenging.
\end{tasks}