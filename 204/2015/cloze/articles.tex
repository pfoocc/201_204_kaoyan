In our contemporary culture, the prospect of communicating with – or even looking at – a stranger is virtually unbearable. Everyone around us seems to agree by the way they cling to their phones, even without a \uline{~~1~~} on a subway.


It's a sad reality – our desire to avoid interacting with other human beings – because there's \uline{~~2~~} to be gained from talking to the stranger standing by you. But you wouldn't know it, \uline{~~3~~} into your phone. This universal protection sends the \uline{~~4~~} : ``Please don't approach me.''


What is it that makes us feel we need to hide \uline{~~5~~} our screens?


One answer is fear, according to Jon Wortmann, an executive mental coach. We fear rejection, or that our innocent social advances will be \uline{~~6~~} as ``weird.'' We fear we'll be \uline{~~7~~} . We fear we'll be disruptive.


Strangers are inherently \uline{~~8~~} to us, so we are more likely to feel \uline{~~9~~} when communicating with them compared with our friends and acquaintances. To avoid this uneasiness, we \uline{~~10~~} to our phones. ``Phones become our security blanket,'' Wortmann says. ``They are our happy glasses that protect us from what we perceive is going to be more \uline{~~11~~} .''


But once we rip off the band-aid, tuck our smartphones in our pockets and look up, it doesn't \uline{~~12~~} so bad. In one 2011 experiment, behavioral scientists Nicholas Epley and Juliana Schroeder asked commuters to do the unthinkable: Start a \uline{~~13~~} . They had Chicago train commuters talk to their fellow \uline{~~14~~} . ``When Dr. Epley and Ms. Schroeder asked other people in the same train station to \uline{~~15~~} how they would feel after talking to a stranger, the commuters thought their \uline{~~16~~} would be more pleasant if they sat on their own,'' The New York Times summarizes. Though the participants didn't expect a positive experience, after they \uline{~~17~~} with the experiment, ``not a single person reported having been embarrassed.''


\uline{~~18~~} , these commutes were reportedly more enjoyable compared with those without communication, which makes absolute sense, \uline{~~19~~} human beings thrive off of social connections. It's that \uline{~~20~~} : Talking to strangers can make you feel connected.