Even in traditional offices, ``the lingua franca of corporate America has gotten much more emotional and much more right-brained than it was 20 years ago,'' said Harvard Business School professor Nancy Koehn. She started spinning off examples. ``If you and I parachuted back to Fortune 500 companies in 1990, we would see much less frequent use of terms like journey, mission, passion. There were goals, there were strategies, there were objectives, but we didn't talk about energy; we didn't talk about passion.''


Koehn pointed out that this new era of corporate vocabulary is very ``team''-oriented – and not by coincidence. ``Let's not forget sports – in male-dominated corporate America, it's still a big deal. It's not explicitly conscious; it's the idea that I'm a coach, and you're my team, and we're in this together. There are lots and lots of CEOs in very different companies, but most think of themselves as coaches and this is their team and they want to win. ''


These terms are also intended to infuse work with meaning – and, as Rakesh Khurana, another professor, points out, increase allegiance to the firm. ``You have the importation of terminology that historically used to be associated with non-profit organizations and religious organizations: terms like vision, values, passion, and purpose,'' said Khurana.


This new focus on personal fulfillment can help keep employees motivated amid increasingly loud debates over work-life balance. The ``mommy wars'' of the 1990s are still going on today, prompting arguments about why women still can't have it all and books like Sheryl Sandberg's Lean In, whose title has become a buzzword in its own right. Terms like unplug, offline, life-hack, bandwidth, and capacity are all about setting boundaries between the office and the home. But if your work is your ``passion,'' you'll be more likely to devote yourself to it, even if that means going home for dinner and then working long after the kids are in bed.


But this seems to be the irony of office speak: Everyone makes fun of it, but managers love it, companies depend on it, and regular people willingly absorb it. As a linguist once said, ``You can get people to think it's nonsense at the same time that you buy into it.'' In a workplace that's fundamentally indifferent to your life and its meaning, office speak can help you figure out how you relate to your work – and how your work defines who you are.


