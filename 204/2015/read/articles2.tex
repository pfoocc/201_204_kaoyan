For years, studies have found that first-generation college students – those who do not have a parent with a college degree – lag other students on a range of education achievement factors. Their grades are lower and their dropout rates are higher. But since such students are most likely to advance economically if they succeed in higher education, colleges and universities have pushed for decades to recruit more of them. This has created ``a paradox'' in that recruiting first-generation students, but then watching many of them fail, means that higher education has ``continued to reproduce and widen, rather than close'' an achievement gap based on social class, according to the depressing beginning of a paper forthcoming in the journal Psychological Science.


But the article is actually quite optimistic, as it outlines a potential solution to this problem, suggesting that an approach (which involves a one-hour, next-to-no-cost program) can close 63 percent of the achievement gap (measured by such factors as grades) between first-generation and other students.


The authors of the paper are from different universities, and their findings are based on a study involving 147 students (who completed the project) at an unnamed private university. First generation was defined as not having a parent with a four-year college degree. Most of the first-generation students (59.1 percent) were recipients of Pell Grants, a federal grant for undergraduates with financial need, while this was true only for 8.6 percent of the students with at least one parent with a four-year degree.


Their thesis – that a relatively modest intervention could have a big impact – was based on the view that first-generation students may be most lacking not in potential but in practical knowledge about how to deal with the issues that face most college students. They cite past research by several authors to show that this is the gap that must be narrowed to close the achievement gap.


Many first-generation students ``struggle to navigate the middle-class culture of higher education, learn the 'rules of the game,' and take advantage of college resources,'' they write. And this becomes more of a problem when colleges don't talk about the class advantages and disadvantages of different groups of students. ``Because US colleges and universities seldom acknowledge how social class can affect students' educational experiences, many first-generation students lack insight about why they are struggling and do not understand how students 'like them' can improve.''