\item Recruiting more first-generation students has \uline{~~~~}.
\begin{tasks}
	\task reduced their dropout rates
	\task narrowed the achievement gap
	\task missed its original purpose
	\task depressed college students
\end{tasks}
\item The authors of the research article are optimistic because \uline{~~~~}.
\begin{tasks}
	\task their findings appeal to students
	\task the recruiting rate has increased
	\task the problem is solvable
	\task their approach is costless
\end{tasks}
\item The study suggests that most first-generation students \uline{~~~~}.
\begin{tasks}
	\task are from single-parent families
	\task study at private universities
	\task are in need of financial support
	\task have failed their college
\end{tasks}
\item The authors of the paper believe that first-generation students \uline{~~~~}.
\begin{tasks}
	\task may lack opportunities to apply for research projects
	\task are inexperienced in handling their issues at college
	\task can have a potential influence on other students
	\task are actually indifferent to the achievement gap
\end{tasks}
\item We may infer from the last paragraph that \uline{~~~~}.
\begin{tasks}
	\task universities often reject the culture of the middle-class
	\task students are usually to blame for their lack of resources
	\task social class greatly helps enrich educational experiences
	\task colleges are partly responsible for the problem in question
\end{tasks}