\item According to Nancy Koehn, office language has become \uline{~~~~}.
\begin{tasks}
	\task less strategic
	\task less energetic
	\task more objective
	\task more emotional
\end{tasks}
\item ``Team''-oriented corporate vocabulary is closely related to \uline{~~~~}.
\begin{tasks}
	\task sports culture
	\task gender difference
	\task historical incidents
	\task athletic executives
\end{tasks}
\item Khurana believes that the importation of terminology aims to \uline{~~~~}.
\begin{tasks}
	\task revive historical terms
	\task promote company image
	\task foster corporate cooperation
	\task strengthen employee loyalty
\end{tasks}
\item It can be inferred that Lean In \uline{~~~~}.
\begin{tasks}
	\task voices for working women
	\task appeals to passionate workaholics
	\task triggers debates among mommies
	\task praises motivated employees
\end{tasks}
\item Which of the following statements is true about office speak?
\begin{tasks}
	\task Linguists believe it to be nonsense.
	\task Regular people mock it but accept it.
	\task Companies find it to be fundamental.
	\task Managers admire it but avoid it.
\end{tasks}