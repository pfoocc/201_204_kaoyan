\item According to Jenny Radesky, digital products are designed to \uline{~~~~}.
\begin{tasks}
	\task simplify routine matters
	\task absorb user attention
	\task better interpersonal relations
	\task increase work efficiency
\end{tasks}
\item Radesky's food-testing exercise shows that mothers' use of devices \uline{~~~~}.
\begin{tasks}
	\task takes away babies' appetite
	\task distracts children's attention
	\task slows down babies' verbal development
	\task reduces mother-child communication
\end{tasks}
\item Radesky cites the ``still face experiment'' to show that \uline{~~~~}.
\begin{tasks}
	\task it is easy for children to get used to blank expressions
	\task verbal expressions are unnecessary for emotional exchange
	\task children are insensitive to changes in their parents' mood
	\task parents need to respond to children's emotional needs
\end{tasks}
\item The oppressive ideology mentioned by Tronick requires parents to \uline{~~~~}.
\begin{tasks}
	\task protect kids from exposure to wild fantasies
	\task teach their kids at least 30,000 words a year
	\task ensure constant interaction with their children
	\task remain concerned about kids' use of screens
\end{tasks}
\item According to Tronick, kids' use of screens may \uline{~~~~}.
\begin{tasks}
	\task give their parents some free time
	\task make their parents more creative
	\task help them with their homework
	\task help them become more attentive
\end{tasks}