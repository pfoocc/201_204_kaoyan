Though often viewed as a problem for western states, the growing frequency of wildfires is a national concern because of its impact on federal tax dollars, says Professor Max Moritz, a specialist in fire ecology and management.


In 2015, the US Forest Service for the first time spent more than half of its \$5.5 billion annual budget fighting fires – nearly double the percentage it spent on such efforts 20 years ago. In effect, fewer federal funds today are going towards the agency's other work – such as forest conservation, watershed and cultural resources management, and infrastructure upkeep – that affect the lives of all Americans.


Another nationwide concern is whether public funds from other agencies are going into construction in fire-prone districts. As Moritz puts it, how often are federal dollars building homes that are likely to be lost to a wildfire?


``It's already a huge problem from a public expenditure perspective for the whole country,'' he says. ``We need to take a magnifying glass to that. Like, 'Wait a minute, is this OK?' Do we want instead to redirect those funds to concentrate on lower-hazard parts of the landscape?''


Such a view would require a corresponding shift in the way US society today views fire, researchers say.


For one thing, conversations about wildfires need to be more inclusive. Over the past decade, the focus has been on climate change – how the warming of the Earth from greenhouse gases is leading to conditions that worsen fires.


While climate is a key element, Moritz says, it shouldn't come at the expense of the rest of the equation.


``The human systems and the landscapes we live on are linked, and the interactions go both ways,'' he says. Failing to recognize that, he notes, leads to ``an overly simplified view of what the solutions might be. Our perception of the problem and of what the solution is becomes very limited.''


At the same time, people continue to treat fire as an event that needs to be wholly controlled and unleashed only out of necessity, says Professor Balch at the University of Colorado. But acknowledging fire's inevitable presence in human life is an attitude crucial to developing the laws, policies, and practices that make it as safe as possible, she says.


``We've disconnected ourselves from living with fire,'' Balch says. ``It is really important to understand and try and tease out what is the human connection with fire today.''