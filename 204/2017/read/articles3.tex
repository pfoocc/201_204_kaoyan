Today, widespread social pressure to immediately go to college in conjunction with increasingly high expectations in a fast-moving world often causes students to completely overlook the possibility of taking a gap year. After all, if everyone you know is going to college in the fall, it seems silly to stay back a year, doesn't it? And after going to school for 12 years, it doesn't feel natural to spend a year doing something that isn't academic.


But while this may be true, it's not a good enough reason to condemn gap years. There's always a constant fear of falling behind everyone else on the socially perpetuated ``race to the finish line,'' whether that be toward graduate school, medical school or a lucrative career. But despite common misconceptions, a gap year does not hinder the success of academic pursuits – in fact, it probably enhances it.


Studies from the United States and Australia show that students who take a gap year are generally better prepared for and perform better in college than those who do not. Rather than pulling students back, a gap year pushes them ahead by preparing them for independence, new responsibilities and environmental changes – all things that first-year students often struggle with the most. Gap year experiences can lessen the blow when it comes to adjusting to college and being thrown into a brand new environment, making it easier to focus on academics and activities rather than acclimation blunders.


If you're not convinced of the inherent value in taking a year off to explore interests, then consider its financial impact on future academic choices. According to the National Center for Education Statistics, nearly 80 percent of college students end up changing their majors at least once. This isn't surprising, considering the basic mandatory high school curriculum leaves students with a poor understanding of the vast academic possibilities that await them in college. Many students find themselves listing one major on their college applications, but switching to another after taking college classes. It's not necessarily a bad thing, but depending on the school, it can be costly to make up credits after switching too late in the game. At Boston College, for example, you would have to complete an extra year were you to switch to the nursing school from another department. Taking a gap year to figure things out initially can help prevent stress and save money later on.