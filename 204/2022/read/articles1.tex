On a recent sunny day, 13,000 chickens roam over Larry Brown's 40 windswept acres in Shiner, Texas. Some rest in the shade of a parked car. Others drink water with the cows. This all seems random, but it's by design, part of what the \$6.1 billion U.S. egg industry bets will be its next big thing: climate-friendly eggs.


These eggs, which are making their debut now on shelves for as much as \$8 a dozen, are still labeled organic and animal-friendly, but they're also from birds that live on farms using regenerative agriculture — special techniques to cultivate rich soils that can trap greenhouse gases. Such eggs could be marketed as helping to fight climate change.


``I'm excited about our progress,'' says Brown, who harvests eggs for Denver-based NestFresh Eggs and is adding more cover crops that draw worms and crickets for the chickens to eat. The birds' waste then fertilizes fields. Such improvements ``allow our hens to forage for higher-quality natural feed that will be good for the land, the hens, and the eggs that we supply to our customers.''


The egg industry's push is the first major test of whether animal products from regenerative farms can become the next premium offering. In barely more than a decade, organic eggs went from being dismissed as a niche product in natural foods stores to being sold at Walmart. More recently there were similar doubts about probiotics and plant-based meats, but both have exploded into major supermarket categories. If the sustainable-egg rollout is successful, it could open the floodgates for regenerative beef, broccoli, and beyond.


Regenerative products could be a hard sell, because the concept is tough to define quickly, says Julie Stanton, associate professor of agricultural economics at Pennsylvania State University Brandywine. Such farming also brings minimal, if any, improvement to the food products (though some producers say their eggs have more protein).


The industry is betting that the same consumers paying more for premium attributes such as free-range, non-GMO, and pasture-raised eggs will embrace sustainability. Surveys show that younger generations are more concerned about climate change, and some of the success of plant-based meat can be chalked up to shoppers wanting to signal their desire to protect the environment. Young adults ``really care about the planet, ''says John Brunnquell, president of Egg Innovations. ``They are absolutely altering the food chain beyond what I think even if they understand what they're doing.'' 