\item The climate-friendly eggs are produced \uline{~~~~}.
\begin{tasks}
	\task at a considerably low cost
	\task at the demand of regular shoppers
	\task as a replacement for organic eggs
	\task on specially designed farms
\end{tasks}
\item Larry Brown is excited about his progress in \uline{~~~~}.
\begin{tasks}
	\task reducing the damage of climate change
	\task accelerating the disposal of wastes
	\task creating a sustainable system
	\task attracting customers to his products
\end{tasks}
\item The example of organic eggs is used in the paragraph 4 to suggest \uline{~~~~}.
\begin{tasks}
	\task the doubts over natural feeds
	\task the setbacks in the eggs industry
	\task the potential of regenerative products
	\task the promotional success of supermarkets
\end{tasks}
\item It can be learned from the paragraph 6 that young people \uline{~~~~}.
\begin{tasks}
	\task are reluctant to change their diet
	\task are likely to buy climate-friendly eggs
	\task are curious about new food
	\task are amazed at agricultural advances
\end{tasks}
\item John Brunnquell would disagree with Julie Stanton over regenerative products' \uline{~~~~}.
\begin{tasks}
	\task nutritional value
	\task standard definition
	\task market prospect
	\task moral implication
\end{tasks}
