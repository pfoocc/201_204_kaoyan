The Internet affords anonymity to its users, a blessing to privacy and freedom of speech. But that very anonymity is also behind the explosion of cyber-crime that has \uline{~~1~~} across the Web.


Can privacy be preserved \uline{~~2~~} bringing safety and security to a world that seems increasingly \uline{~~3~~} ?


Last month, Howard Schmidt, the nation's cyber-czar, offered the federal government a \uline{~~4~~} to make the Web a safer place – a ``voluntary trusted identity'' system that would be the high-tech \uline{~~5~~} of a physical key, a fingerprint and a photo ID card, all rolled \uline{~~6~~} one. The system might use a smart identity card, or a digital credential \uline{~~7~~} to a specific computer, and would authenticate users at a range of online services.


The idea is to \uline{~~8~~} a federation of private online identity systems. Users could \uline{~~9~~} which system to join, and only registered users whose identities have been authenticated could navigate those systems. The approach contrasts with one that would require an Internet driver's license \uline{~~10~~} by the government.


Google and Microsoft are among companies that already have these ``single sign-on'' systems that make it possible for users to \uline{~~11~~} just once but use many different services.


\uline{~~12~~} , the approach would create a ``walled garden'' in cyberspace, with safe ``neighborhoods'' and bright ``streetlights'' to establish a sense of a \uline{~~13~~} community.


Mr. Schmidt described it as a ``voluntary ecosystem'' in which ``individuals and organizations can complete online transactions with \uline{~~14~~} , trusting the identities of each other and the identities of the infrastructure \uline{~~15~~} which the transaction runs.''


Still, the administration's plan has \uline{~~16~~} privacy rights activists. Some applaud the approach; others are concerned. It seems clear that such a scheme is an initiative push toward what would \uline{~~17~~} be a compulsory Internet ``driver's license'' mentality.


The plan has also been greeted with \uline{~~18~~} by some computer security experts, who worry that the ``voluntary ecosystem'' envisioned by Mr. Schmidt would still leave much of the Internet \uline{~~19~~} . They argue that all Internet users should be \uline{~~20~~} to register and identify themselves, in the same way that drivers must be licensed to drive on public roads.