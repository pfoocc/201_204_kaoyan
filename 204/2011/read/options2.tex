\item By saying ``Newspapers like \uline{~~~~}... their own doom'' (Lines3-4, Para.1), the author indicates that newspapers \uline{~~~~}.
\begin{tasks}
	\task neglected the sign of crisis
	\task failed to get state subsidies
	\task were not charitable corporations
	\task were in a desperate situation
\end{tasks}
\item Some newspapers refused delivery to distant suburbs probably because \uline{~~~~}.
\begin{tasks}
	\task readers threatened to pay less
	\task newspapers wanted to reduce costs
	\task journalists reported little about these areas
	\task subscribers complained about slimmer products
\end{tasks}
\item Compared with their American counterparts, Japanese newspapers are much more stable because they \uline{~~~~}.
\begin{tasks}
	\task have more sources of revenue
	\task have more balanced newsrooms
	\task are less dependent on advertising
	\task are less affected by readership
\end{tasks}
\item What can be inferred from the last paragraph about the current newspaper business?
\begin{tasks}
	\task Distinctiveness is an essential feature of newspapers.
	\task Completeness is to blame for the failure of newspaper.
	\task Foreign bureaus play a crucial role in the newspaper business.
	\task Readers have lost their interest in car and film reviews.
\end{tasks}
\item The most appropriate title for this text would be \uline{~~~~}.
\begin{tasks}
	\task American Newspapers: Struggling for Survival
	\task American Newspapers: Gone with the Wind
	\task American Newspapers: A Thriving Business
	\task American Newspapers: A Hopeless Story
\end{tasks}