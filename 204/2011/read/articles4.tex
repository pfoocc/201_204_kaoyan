Will the European Union make it? The question would have sounded strange not long ago. Now even the project's greatest cheerleaders talk of a continent facing a ``Bermuda triangle'' of debt, population decline and lower growth.


As well as those chronic problems, the EU faces an acute crisis in its economic core, the 16 countries that use the single currency. Markets have lost faith that the euro zone's economies, weaker or stronger, will one day converge thanks to the discipline of sharing a single currency, which denies uncompetitive members the quick fix of devaluation.


Yet the debate about how to save Europe's single currency from disintegration is stuck. It is stuck because the euro zone's dominant powers, France and Germany, agree on the need for greater harmonisation within the euro zone, but disagree about what to harmonise.


Germany thinks the euro must be saved by stricter rules on borrowing, spending and competitiveness, backed by quasi-automatic sanctions for governments that do not obey. These might include threats to freeze EU funds for poorer regions and EU mega-projects, and even the suspension of a country's voting rights in EU ministerial councils. It insists that economic co-ordination should involve all 27 members of the EU club, among whom there is a small majority for free-market liberalism and economic rigour; in the inner core alone, Germany fears, a small majority favour French interference.


A ``southern'' camp headed by France wants something different: ``European economic government'' within an inner core of euro-zone members. Translated, that means politicians intervening in monetary policy and a system of redistribution from richer to poorer members, via cheaper borrowing for governments through common Eurobonds or complete fiscal transfers. Finally, figures close to the Franch government have murmured, euro-zone members should agree to some fiscal and social harmonisation: e.g., curbing competition in corporate-tax rates or labour costs.


It is too soon to write off the EU. It remains the world's largest trading block.  At its best, the European project is remarkably liberal: built around a single market of 27 rich and poor countries, its internal borders are far more open to goods, capital and labour than any comparable trading area. It is an ambitious attempt to blunt the sharpest edges of globalisation, and make capitalism benign.


