\item The postwar American housing style largely reflected the Americans' \uline{~~~~}.
\begin{tasks}
	\task prosperity and growth
	\task efficiency and practicality
	\task restraint and confidence
	\task pride and faithfulness
\end{tasks}
\item Which of the following can be inferred from Paragraph 3 about the Bauhaus?
\begin{tasks}
	\task It was founded by Ludwig Mies van der Rohe.
	\task Its designing concept was affected by World War II.
	\task Most American architects used to be associated with it.
	\task It had a great influence upon American architecture.
\end{tasks}
\item Mies held that elegance of architectural design \uline{~~~~}.
\begin{tasks}
	\task was related to large space
	\task was identified with emptiness
	\task was not reliant on abundant decoration
	\task was not associated with efficiency
\end{tasks}
\item What is true about the apartments Mies built on Chicago's Lake Shore Drive?
\begin{tasks}
	\task They ignored details and proportions.
	\task They were built with materials popular at that time.
	\task They were more spacious than neighboring buildings.
	\task They shared some characteristics of abstract art.
\end{tasks}
\item What can we learn about the design of the ``Case Study Houses''?
\begin{tasks}
	\task Mechanical devices were widely used.
	\task Natural scenes were taken into consideration.
	\task Details were sacrificed for the overall effect.
	\task Eco-friendly materials were employed.
\end{tasks}
