\item The EU is faced with so many problems that \uline{~~~~}.
\begin{tasks}
	\task it has more or less lost faith in markets
	\task even its supporters begin to feel concerned
	\task some of its member countries plan to abandon euro
	\task it intends to deny the possibility of devaluation
\end{tasks}
\item The debate over the EU's single currency is stuck because the dominant powers \uline{~~~~}.
\begin{tasks}
	\task are competing for the leading position
	\task are busy handling their own crises
	\task fail to reach an agreement on harmonisation
	\task disagree on the steps towards disintegration
\end{tasks}
\item To solve the euro problem, Germany proposed that \uline{~~~~}.
\begin{tasks}
	\task EU funds for poor regions be increased
	\task stricter regulations be imposed
	\task only core members be involved in economic co-ordination
	\task voting rights of the EU members be guaranteed
\end{tasks}
\item The French proposal of handling the crisis implies that \uline{~~~~}.
\begin{tasks}
	\task poor countries are more likely to get funds
	\task strict monetary policy will be applied to poor countries
	\task loans will be readily available to rich countries
	\task rich countries will basically control Eurobonds
\end{tasks}
\item Regarding the future of the EU, the author seems to feel \uline{~~~~}.
\begin{tasks}
	\task pessimistic
	\task desperate
	\task conceited
	\task hopeful
\end{tasks}