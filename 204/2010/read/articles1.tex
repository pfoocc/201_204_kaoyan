The longest bull run in a century of art-market history ended on a dramatic note with a sale of 56 works by Damien Hirst, Beautiful Inside My Head Forever, at Sotheby's in London on September 15th 2008. All but two pieces sold, fetching more than £70m, a record for a sale by a single artist. It was a last victory. As the auctioneer called out bids, in New York one of the oldest banks on Wall Street, Lehman Brothers, filed for bankruptcy.


The world art market had already been losing momentum for a while after rising bewilderingly since 2003. At its peak in 2007 it was worth some \$65 billion, reckons Clare McAndrew, founder of Arts Economics, a research firm – double the figure five years earlier. Since then it may have come down to \$50 billion. But the market generates interest far beyond its size because it brings together great wealth, enormous egos, greed, passion and controversy in a way matched by few other industries.


In the weeks and months that followed Mr. Hirst's sale, spending of any sort became deeply unfashionable. In the art world that meant collectors stayed away from galleries and salerooms. Sales of contemporary art fell by two-thirds, and in the most overheated sector, they were down by nearly 90\% in the year to November 2008. Within weeks the world's two biggest auction houses, Sotheby's and Christie's, had to pay out nearly \$200m in guarantees to clients who had placed works for sale with them.


The current downturn in the art market is the worst since the Japanese stopped buying Impressionists at the end of 1989. This time experts reckon that prices are about 40% down on their peak on average, though some have been far more fluctuant. But Edward Dolman, Christie's chief executive, says: ``I'm pretty confident we're at the bottom.''


What makes this slump different from the last, he says, is that there are still buyers in the market. Almost everyone who was interviewed for this special report said that the biggest problem at the moment is not a lack of demand but a lack of good work to sell. The three Ds – death, debt and divorce – still deliver works of art to the market. But anyone who does not have to sell is keeping away, waiting for confidence to return.