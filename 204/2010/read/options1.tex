\item In the first paragraph, Damien Hirst's sale was referred to as ``a last victory'' because \uline{~~~~}.
\begin{tasks}
	\task the art market had witnessed a succession of victories
	\task the auctioneer finally got the two pieces at the highest bids
	\task Beautiful Inside My Head Forever won over all masterpieces
	\task it was successfully made just before the world financial crisis
\end{tasks}
\item By saying ``spending of any sort became deeply unfashionable'' (Line 1-2, Para. 3), the author suggests that \uline{~~~~}.
\begin{tasks}
	\task collectors were no longer actively involved in art-market auctions
	\task people stopped every kind of spending and stayed away from galleries
	\task art collection as a fashion had lost its appeal to a great extent
	\task works of art in general had gone out of fashion so they were not worth buying
\end{tasks}
\item Which of the following statements is NOT true?
\begin{tasks}
	\task Sales of contemporary art fell dramatically from 2007 to 2008.
	\task The art market surpassed many other industries in momentum.
	\task The art market generally went downward in various ways.
	\task Some art dealers were awaiting better chances to come.
\end{tasks}
\item The three Ds mentioned in the last paragraph are \uline{~~~~}.
\begin{tasks}
	\task auction houses' favorites
	\task contemporary trends
	\task factors promoting artwork circulation
	\task styles representing Impressionists
\end{tasks}
\item The most appropriate title for this text could be \uline{~~~~}.
\begin{tasks}
	\task Fluctuation of Art Prices
	\task Up-to-date Art Auctions
	\task Art Market in Decline
	\task Shifted Interest in Arts
\end{tasks}
