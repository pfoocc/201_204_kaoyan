\item From the principles of the US jury system, we learn that \uline{~~~~}.
\begin{tasks}
	\task both literate and illiterate people can serve on juries
	\task defendants are immune from trial by their peers
	\task no age limit should be imposed for jury service
	\task judgment should consider the opinion of the public
\end{tasks}
\item The practice of selecting so-called elite jurors prior to 1968 showed \uline{~~~~}.
\begin{tasks}
	\task the inadequacy of antidiscrimination laws
	\task the prevalent discrimination against certain races
	\task the conflicting ideals in jury selection procedures
	\task the arrogance common among the Supreme Court judges
\end{tasks}
\item Even in the 1960s, women were seldom on the jury list in some states because.
\begin{tasks}
	\task they were automatically banned by state laws
	\task they fell far short of the required qualifications
	\task they were supposed to perform domestic duties
	\task they tended to evade public engagement
\end{tasks}
\item After the Jury Selection and Service Act was passed, \uline{~~~~}.
\begin{tasks}
	\task sex discrimination in jury selection was unconstitutional and had to be abolished
	\task educational requirements became less rigid in the selection of federal jurors
	\task jurors at the state level ought to be representative of the entire community
	\task states ought to conform to the federal court in reforming the jury system
\end{tasks}
\item In discussing the US jury system, the text centers on \uline{~~~~}.
\begin{tasks}
	\task its nature and problems
	\task its characteristics and tradition
	\task its problems and their solutions
	\task its tradition and development
\end{tasks}