Given the advantages of electronic money, you might think that we would move quickly to the cashless society in which all payments are made electronically. \uline{~~1~~} , a true cashless society is probably not around the corner. Indeed, predictions have been \uline{~~2~~} for two decades but have not yet come to fruition. For example, Business Week predicted in 1975 that electronic means of payment would soon ``revolutionize the very \uline{~~3~~} of money itself,'' only to \uline{~~4~~} itself several years later. Why has the movement to a cashless society been so \uline{~~5~~} in coming?


Although electronic means of payment may be more efficient than a payments system based on paper, several factors work \uline{~~6~~} the disappearance of the paper system. First, it is very \uline{~~7~~} to set up the computer, card reader, and telecommunications networks necessary to make electronic money the \uline{~~8~~} form of payment. Second, paper checks have the advantage that they \uline{~~9~~} receipts, something that many consumers are unwilling to \uline{~~10~~} . Third, the use of paper checks gives consumers several days of ``float'' – it takes several days \uline{~~11~~} a check is cashed and funds are \uline{~~12~~} from the issuer's account, which means that the writer of the check can earn interest on the funds in the meantime. \uline{~~13~~} electronic payments arc immediate, they eliminate the float for the consumer.


Fourth, electronic means of payment may \uline{~~14~~} security and privacy concerns. We often hear media reports that an unauthorized hacker has been able to access a computer database and to alter information \uline{~~15~~} there. The fact that this is not an \uline{~~16~~} occurrence means that dishonest persons might be able to access bank accounts in electronic payments systems and \uline{~~17~~} from someone else's accounts. The \uline{~~18~~} of this type of fraud is no easy task, and a new field of computer science is developing to \uline{~~19~~} security issues. A further concern is that the use of electronic means of payment leaves an electronic \uline{~~20~~} that contains a large amount of personal data. There are concerns that government, employers, and marketers might be able to access these data, thereby violating our privacy.