Europe is not a gender-equality heaven. In particular, the corporate workplace will never be completely family-friendly until women are part of senior management decisions, and Europe's top corporate-governance positions remain overwhelmingly male. Indeed, women hold only 14 per cent of positions on European corporate boards.


The European Union is now considering legislation to compel corporate boards to maintain a certain proportion of women – up to 60 per cent. This proposed mandate was born of frustration. Last year, European Commission Vice President Viviane Reding issued a call to voluntary action. Reding invited corporations to sign up for gender balance goals of 40 per cent female board membership. But her appeal was considered a failure: only 24 companies took it up.


Do we need quotas to ensure that women can continue to climb the corporate ladder fairly as they balance work and family?


``Personally, I don't like quotas,'' Reding said recently. ``But I like what the quotas do.'' Quotas get action: they ``open the way to equality and they break through the glass ceiling,'' according to Reding, a result seen in France and other countries with legally binding provisions on placing women in top business positions.


I understand Reding's reluctance – and her frustration. I don't like quotas either; they run counter to my belief in meritocracy, governance by the capable. But, when one considers the obstacles to achieving the meritocratic ideal, it does look as if a fairer world must be temporarily ordered.


After all, four decades of evidence has now shown that corporations in Europe as well as the US are evading the meritocratic hiring and promotion of women to top positions – no matter how much ``soft pressure'' is put upon them. When women do break through to the summit of corporate power – as, for example, Sheryl Sandberg recently did at Facebook – they attract massive attention precisely because they remain the exception to the rule.


If appropriate public policies were in place to help all women – whether CEOs or their children's caregivers – and all families, Sandberg would be no more newsworthy than any other highly capable person living in a more just society.


