Scientists have found that although we are prone to snap overreactions, if we take a moment and think about how we are likely to react, we can reduce or even eliminate the negative effects of our quick, hard-wired responses.


Snap decisions can be important defense mechanisms; if we are judging whether someone is dangerous, our brains and bodies are hard-wired to react very quickly, within milliseconds. But we need more time to assess other factors. To accurately tell whether someone is sociable, studies show, we need at least a minute, preferably five. It takes a while to judge complex aspects of personality, like neuroticism or open -mindedness.


But snap decisions in reaction to rapid stimuli aren't exclusive to the interpersonal realm. Psychologists at the University of Toronto found that viewing a fast-food logo for just a few milliseconds primes us to read 20 percent faster, even though reading has little to do with eating. We unconsciously associate fast food with speed and impatience and carry those impulses into whatever else we're doing, Subjects exposed to fast-food flashes also tend to think a musical piece lasts too long.


Yet we can reverse such influences. If we know we will overreact to consumer products or housing options when we see a happy face (one reason good sales representatives and real estate agents are always smiling), we can take a moment before buying. If we know female job screeners are more likely to reject attractive female applicants, we can help screeners understand their biases – or hire outside screeners.


John Gottman, the marriage expert, explains that we quickly ``thin slice'' information reliably only after we ground such snap reactions in ``thick sliced'' long-term study. When Dr. Gottman really wants to assess whether a couple will stay together, he invites them to his island retreat for a much longer evaluation: two days, not two seconds.


Our ability to mute our hard-wired reactions by pausing is what differentiates us from animals: dogs can think about the future only intermittently or for a few minutes. But historically we have spent about 12 percent of our days contemplating the longer term. Although technology might change the way we react, it hasn't changed our nature. We still have the imaginative capacity to rise above temptation and reverse the high-speed trend.


