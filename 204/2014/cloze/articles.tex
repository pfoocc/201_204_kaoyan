Thinner isn't always better. A number of studies have \uline{~~1~~} that normal-weight people are in fact at higher risk of some diseases compared to those who are overweight. And there are health conditions for which being overweight is actually \uline{~~2~~} . For example, heavier women are less likely to develop calcium deficiency than thin women. \uline{~~3~~} , among the elderly, being somewhat overweight is often an \uline{~~4~~} of good health.


Of even greater \uline{~~5~~} is the fact that obesity turns out to be very difficult to define. It is often defined \uline{~~6~~} body mass index, or BMI. BMI \uline{~~7~~} body mass divided by the square of height. An adult with a BMI of 18 to 25 is often considered to be normal weight. Between 25 and 30 is overweight. And over 30 is considered obese. Obesity, \uline{~~8~~} , can be divided into moderately obese, severely obese, and very severely obese.


While such numerical standards seem \uline{~~9~~} , they are not. Obesity is probably less a matter of weight than body fat. Some people with a high BMI are in fact extremely fit, \uline{~~10~~} others with a low BMI may be in poor \uline{~~11~~} . For example, many collegiate and professional football players \uline{~~12~~} as obese, though their percentage body fat is low. Conversely, someone with a small frame may have high body fat but a \uline{~~13~~} BMI.


Today we have a(n) \uline{~~14~~} to label obesity as a disgrace. The overweight are sometimes \uline{~~15~~} in the media with their faces covered. Stereotypes \uline{~~16~~} with obesity include laziness, lack of will power, and lower prospects for success. Teachers, employers, and health professionals have been shown to harbor biases against the obese. \uline{~~17~~} very young children tend to look down on the overweight, and teasing about body build has long been a problem in schools.


Negative attitudes toward obesity, \uline{~~18~~} in health concerns, have stimulated a number of anti-obesity \uline{~~19~~} . My own hospital system has banned sugary drinks from its facilities. Many employers have instituted weight loss and fitness initiatives. Michelle Obama has launched a high-visibility campaign \uline{~~20~~} childhood obesity, even claiming that it represents our greatest national security threat.