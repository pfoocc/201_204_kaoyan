Homework has never been terribly popular with students and even many parents, but in recent years it has been particularly scorned. School districts across the country, most recently Los Angeles Unified, are revising their thinking on this educational ritual. Unfortunately, L.A. Unified has produced an inflexible policy which mandates that with the exception of some advanced courses, homework may no longer count for more than 10% of a student's academic grade.


This rule is meant to address the difficulty that students from impoverished or chaotic homes might have in completing their homework. But the policy is unclear and contradictory. Certainly, no homework should be assigned that students cannot complete on their own or that they cannot do without expensive equipment. But if the district is essentially giving a pass to students who do not do their homework because of complicated family lives, it is going riskily close to the implication that standards need to be lowered for poor children.


District administrators say that homework will still be a part of schooling; teachers are allowed to assign as much of it as they want. But with homework counting for no more than 10% of their grades, students can easily skip half their homework and see very little difference on their report cards. Some students might do well on state tests without completing their homework, but what about  the students who performed well on the tests and did their homework? It is quite possible that the homework helped. Yet rather than empowering teachers to find what works best for their students, the policy imposes a flat, across-the-board rule.


At the same time, the policy addresses none of the truly thorny questions about homework. If the district finds homework to be unimportant to its students' academic achievement, it should move to reduce or eliminate the assignments, not make them count for almost nothing. Conversely, if homework matters, it should account for a significant portion of the grade. Meanwhile, this policy does nothing to ensure that the homework students receive is meaningful or appropriate to their age and the subject, or that teachers are not assigning more than they are willing to review and correct.


The homework rules should be put on hold while the school board, which is responsible for setting educational policy, looks into the matter and conducts public hearings. It is not too late for L.A. Unified to do homework right.


