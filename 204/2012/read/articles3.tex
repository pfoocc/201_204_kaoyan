In 2010, a federal judge shook America's biotech industry to its core. Companies had won patents for isolated DNA for decades – by 2005 some 20% of human genes were patented. But in March 2010 a judge ruled that genes were unpatentable. Executives were violently agitated. The Biotechnology Industry Organisation (BIO), a trade group, assured members that this was just a ``preliminary step'' in a longer battle.


On July 29th they were relieved, at least temporarily. A federal appeals court overturned the prior decision, ruling that Myriad Genetics could indeed hold patents to two genes that help forecast a woman's risk of breast cancer. The chief executive of Myriad, a company in Utah, said the ruling was a blessing to firms and patients alike.


But as companies continue their attempts at personalised medicine, the courts will remain rather busy. The Myriad case itself is probably not over. Critics make three main arguments against gene patents: a gene is a product of nature, so it may not be patented; gene patents suppress innovation rather than reward it; and patents' monopolies restrict access to genetic tests such as Myriad's. A growing number seem to agree. Last year a federal task-force urged reform for patents related to genetic tests. In October the Department of Justice filed a brief in the Myriad case, arguing that an isolated DNA molecule ``is no less a product of nature…than are cotton fibres that have been separated from cotton seeds.''


Despite the appeals court's decision, big questions remain unanswered. For example, it is unclear whether the sequencing of a whole genome violates the patents of individual genes within it. The case may yet reach the Supreme Court.


As the industry advances, however, other suits may have an even greater impact. Companies are unlikely to file many more patents for human DNA molecules – most are already patented or in the public domain. Firms are now studying how genes interact, looking for correlations that might be used to determine the causes of disease or predict a drug's efficacy. Companies are eager to win patents for ``connecting the dots,'' explains Hans Sauer, a lawyer for the BIO.


Their success may be determined by a suit related to this issue, brought by the Mayo Clinic, which the Supreme Court will hear in its next term. The BIO recently held a convention which included sessions to coach lawyers on the shifting landscape for patents. Each meeting was packed.