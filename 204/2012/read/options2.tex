\item By saying ``it is…the rainbow'' (Line 3, Para. 1), the author means pink \uline{~~~~}.
\begin{tasks}
	\task cannot explain girls' lack of imagination
	\task should not be associated with girls' innocence
	\task should not be the sole representation of girlhood
	\task cannot influence girls' lives and interests
\end{tasks}
\item According to Paragraph 2, which of the following is true of colours?
\begin{tasks}
	\task Colours are encoded in girls' DNA.
	\task Blue used to be regarded as the colour for girls.
	\task White is preferred by babies.
	\task Pink used to be a neutral colour in symbolising genders.
\end{tasks}
\item The author suggests that our perception of children's psychological development was much influenced by \uline{~~~~}.
\begin{tasks}
	\task the observation of children's nature
	\task the marketing of products for children
	\task researches into children's behaviour
	\task studies of childhood consumption
\end{tasks}
\item We may learn from Paragraph 4 that department stores were advised to \uline{~~~~}.
\begin{tasks}
	\task classify consumers into smaller groups
	\task attach equal importance to different genders
	\task focus on infant wear and older kids' clothes
	\task create some common shoppers' terms
\end{tasks}
\item It can be concluded that girls' attraction to pink seems to be \uline{~~~~}.
\begin{tasks}
	\task fully understood by clothing manufacturers
	\task clearly explained by their inborn tendency
	\task mainly imposed by profit-driven businessmen
	\task well interpreted by psychological experts
\end{tasks}