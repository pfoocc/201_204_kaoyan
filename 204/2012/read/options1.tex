\item It is implied in Paragraph 1 that nowadays homework \uline{~~~~}.
\begin{tasks}
	\task is receiving more criticism
	\task is gaining more preferences
	\task is no longer an educational ritual
	\task is not required for advanced courses
\end{tasks}
\item L.A. Unified has made the rule about homework mainly because poor students.
\begin{tasks}
	\task tend to have moderate expectations for their education
	\task have asked for a different educational standard
	\task may have problems finishing their homework
	\task have voiced their complaints about homework
\end{tasks}
\item According to Paragraph 3, one problem with the policy is that it may \uline{~~~~}.
\begin{tasks}
	\task result in students' indifference to their report cards
	\task undermine the authority of state tests
	\task restrict teachers' power in education
	\task discourage students from doing homework
\end{tasks}
\item As mentioned in Paragraph 4, a key question unanswered about homework is whether \uline{~~~~}.
\begin{tasks}
	\task it should be eliminated
	\task it counts much in schooling
	\task it places extra burdens on teachers
	\task it is important for grades
\end{tasks}
\item A suitable title for this text could be \uline{~~~~}.
\begin{tasks}
	\task A Faulty Approach to Homework
	\task A Welcomed Policy for Poor Students
	\task Thorny Questions about Homework
	\task Wrong Interpretations of an Educational Policy
\end{tasks}