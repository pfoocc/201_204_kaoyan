\item By saying ``to find silver linings'' (Line 1, Para. 2) the author suggests that the jobless try to \uline{~~~~}.
\begin{tasks}
	\task seek subsidies from the government
	\task make profits from the troubled economy
	\task explore reasons for the unemployment
	\task look on the bright side of the recession
\end{tasks}
\item According to Paragraph 2, the recession has made people \uline{~~~~}.
\begin{tasks}
	\task struggle against each other
	\task realize the national dream
	\task challenge their prudence
	\task reconsider their lifestyle
\end{tasks}
\item Benjamin Friedman believes that economic recessions may \uline{~~~~}.
\begin{tasks}
	\task impose a heavier burden on immigrants
	\task bring out more evils of human nature
	\task promote the advance of rights and freedoms
	\task ease conflicts between races and classes
\end{tasks}
\item The research of Till Von Wachter suggests that in the recession graduates from elite universities tend to \uline{~~~~}.
\begin{tasks}
	\task lag behind the others due to decreased opportunities
	\task catch up quickly with experienced employees
	\task see their life chances as dimmed as the others'
	\task recover more quickly than the others
\end{tasks}
\item The author thinks that the influence of hard times on society is \uline{~~~~}.
\begin{tasks}
	\task trivial
	\task positive
	\task certain
	\task destructive
\end{tasks}