\item Quinn and her colleagues conducted a test to see if rats can \uline{~~~~}.
\begin{tasks}
	\task pick up social signals from non-living rats
	\task distinguish a friendly rat from a hostile one
	\task attain sociable traits through special training
	\task send out warning messages to their fellows
\end{tasks}
\item What did the asocial robot do during the experiment?
\begin{tasks}
	\task It followed the social robot.
	\task It played with some toys.
	\task It set the trapped rats free.
	\task It moved around alone.
\end{tasks}
\item According to Quinn, the rats released the social robot because they \uline{~~~~}.
\begin{tasks}
	\task tried to practice a means of escape.
	\task expected it to do the same in return.
	\task wanted to display their intelligence.
	\task considered that an interesting game.
\end{tasks}
\item Janet Wiles notes that rats \uline{~~~~}.
\begin{tasks}
	\task can remember other rats' facial features.
	\task differentiate smells better than sizes.
	\task respond more to actions than to looks.
	\task can be scared by a plastic box on wheels.
\end{tasks}
\item It can be learned from the text that rats \uline{~~~~}.
\begin{tasks}
	\task appear to be adaptable to new surroundings
	\task are more socially active than other animals
	\task behave differently from children in socializing
	\task are more sensitive to social cues than expected
\end{tasks}