Rats and other animals need to be highly attuned to social signals from others so they can identify friends to cooperate with and enemies to avoid. To find out if this extends to non-living beings, Laleh Quinn at the University of California, San Diego, and her colleagues tested whether rats can detect social signals from robotic rats.


They housed eight adult rats with two types of robotic rat—one social and one asocial—for four days. The robot rats were quite minimalist, resembling a chunkier version of a computer mouse with wheels to move around and colourful markings.


During the experiment, the social robot rat followed the living rats around, played with the same toys, and opened cage doors to let trapped rats escape. Meanwhile, the asocial robot simply moved forwards and backwards and side to side.


Next, the researchers trapped the robots in cages and gave the rats the opportunity to release them by pressing a lever. Across 18 trials each, the living rats were 52 per cent more likely on average to set the social robot free than the asocial one. This suggests that the rats perceived the social robot as a genuine social being, says Quinn. The rats may have bonded more with the social robot because it displayed behaviors like communal exploring and playing. This could lead to the rats better remembering having freed it earlier, and wanting the robot to return the favour when they get trapped, she says.


``Rats have been shown to engage in multiple forms of reciprocal help and cooperation, including what is referred to as direct reciprocity where a rat will help another rat that has previously helped them,'' says Quinn.


The readiness of the rats to befriend the social robot was surprising given its minimal design. The robot was the same size as a regular rat but resembled a simple plastic box on wheels. `` We'd assumed we'd have to give it a moving head and tail, facial features, and put a scent on it to make it smell like a real rat, but that wasn't necessary,''says Janet Wiles at the University of Queensland in Australia, who helped with the research.


The finding shows how sensitive rats are to social cues, even when they come from basic robots. says Wiles. Similarly, children tend to treat robots as if they are fellow beings, even when they display only simple social signals. ``We humans seem to be fascinated by robots, and it turns out other animals are too,'' says Wiles.


