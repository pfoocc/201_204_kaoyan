The Internet may be changing merely what we remember, not our capacity to do so, suggests Columbia University psychology professor Betsy Sparrow. In 2011, Sparrow led a study in which participants were asked to record 40 factoids in a computer (``an ostrich's eye is bigger than its brain,'' for example). Half of the participants were told the information would be erased, while the other half were told it would be saved. Guess what? The latter group made no effort to recall the information when quizzed on it later, because they knew they could find it on their computers. In the same study, a group was asked to remember both the information and the folders it was stored in. They didn't remember the information, but they remembered how to find the folders. In other words, human memory is not deteriorating but ``adapting to new communications technology,'' Sparrow says.


In a very practical way, the Internet is becoming an external hard drive for our memories, a process known as ``cognitive offloading.'' Traditionally, this role was fulfilled by data banks, libraries, and other humans. Your father may never remember birthdays because your mother does, for instance. Some worry that this is having a destructive effect on society but Sparrow sees an upside. Perhaps, she suggests, the trend will change our approach to learning from a focus on individual facts and memorization to an emphasis on more conceptual thinking - something that is not available on the Internet. ``I personally have never seen all that much intellectual value in memorizing things,'' Sparrow says, adding that we haven't lost our ability to do it.


Still other experts say it's too soon to understand how the Internet affects our brains. There is no experimental evidence showing that it interferes with our ability to focus, for instance, wrote psychologists Christopher Chabris and Daniel J. Simons. And surfing the web exercised the brain more than reading did among computer-savvy older adults in a 2008 study involving 24 participants at the Semel Institute for Neuroscience and Human Behavior at the University of California, Los Angeles.


``There may be costs associated with our increased reliance on the Internet, but I'd have to imagine that overall the benefits are going to outweigh those costs,'' observes psychology professor Benjamin Storm. ``It seems pretty clear that memory is changing, but is it changing for the better? At this point, we don't know.''