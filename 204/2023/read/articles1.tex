In the quest for the perfect lawns, homeowners across the country are taking a shortcut -- and it is the environment that is paying the price. About eight million square meters of plastic grass is sold each year but oppositions has now spread to the highest gardening circles. The Chelsen Flower Show has banned fake grass from this year's event, declaiming it to be not part of its ethos. The Royal Horticultural Society (RHS), which norms the annual show in west London, says it has introduced the ban because of the damage plastic grass does to the environment and biodiversity.


Ed Horne of the RHS said: we launched our sustainability strategy last year and fake grass is just not in line with our ethos and views on plastic. We recommend using real grass because of its environment benefits, which include supporting wildlife, alleviating flooding and cooling the environment.


The RHS's decision comes as compaginers try to raise awareness of the problem fake grass cause. A Twitter account, which claims to ``cut through the greenwash'' of artificial grass, already has more than 20,000 followers. It is trying to encourage people to sigh two petitions, one calling for a ban on the sale of plastic grass and another calling for an ``ecological damage'' tax on such lawns. They have gathered 7,276 and 11,282 signatures.


However, supporters of fake grass point out that there's also an environmental impact with natural lawns, which need mowing and therefore usually consume electricity or petrol. The industry also points out that real grass require considerable amounts of water, weed killer or other treatments and that people who lay fake grass tend to use their garden more. The industry also claims that people who lay fake grass spend on average of £500 trees or shrouds for their garden, which provides habitat for insects.


In response to another petition last year about banning fake lawns, which gathered 30 , 000 signatures , the government responded that it has ``no plans to ban the use of artificial grass.''


It added: ``We prefer to help people and organizations make the right choice rather than legislating on such matters. However, the use of artificial grass must comply with the legal and policy safeguards in place to protect biodiversity and ensure sustainable drainage, while measures such as the strengthened biodiversity duty should serve to encourage public authorities to consider sustainable alternatives.'' 