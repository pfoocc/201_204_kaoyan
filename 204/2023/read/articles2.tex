It's easy to dismiss as absurd the federal government's ideas for plugging the chronic funding gap of our national parks. Can anyone really think it's a good idea to allow Amazon deliveries to your tent in Yosemite or food trucks to line up under the redwood trees at Sequoia National Park?


But the govemment is right about one thing: U.S. national parks are in crisis. Collectively, they have a maintenance backlog of more than \$12 bllion. Roads, trails, restrooms, visitor centers and other infrastructure are crumbling.


But privatizing and commercializing the campgrounds would not be a crue-all. Campgrounds are a tiny portion of the overall infrastructure backlog, and businesses in the parks hand over, on average, only about 5\% of their revenues to the National Park Service.


Moreover, increased privatization would certainly undercut one of the major reasons why 300 million visitors come to the parks each year: to enjoy nature and get a break from the commercial drumbeat that overwhelms daily life.


The real problem is that the parks have been chronically starved of funding. An economic survey of 700 U.S. taxpayers foundthat people would be willing to pay a significant amount of money to make sure the parks and their programs are kept intact. Some 81\% ofrespondentsaid they would be willing to pay addítional taxes for the next 10 years to avoid anycuts to the national parks.


The natiopal parks provide greaf yaluejto U.S. residents both as places to escape and assymbols of nature. On top of this, they produce value from their extensive educational programs, their positive impact on the climate through carbon sequestration, their contribution to our cultural and artistic life, and of course through tourism. The parks also help keep America's past alive, working with thousands of local jurisdictions around the country to protect historical sites and to bring the stories of these places to life.


The parks do all this on a shoestring. Congress allocates only 3 bilion a year to the national park system — an amount that has been flat since 2001 (in inflation-adusted dollars) with the exception of a onetime boost in 2009. Meanwhile, the number of annual visitors has increased more than 50\% since 1980, and now stands at 330 million visitors per year.