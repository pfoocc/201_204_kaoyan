It's not difficult to set targets for staff. It is much harder, \uline{~~1~~} , to understand their negative consequences. Most work-related behaviors have multiple components. \uline{~~2~~} one and the others become distorted.


Travel on a London bus and you'll \uline{~~3~~} see how this works with drivers. Watch people get on and show their tickets. Are they carefully inspected? Never. Do people get on without paying? Of course! Are there inspectors to \uline{~~4~~} that people have paid? Possibly, but very few. And people who run for the bus? They are \uline{~~5~~} . How about jumping lights? Buses do so almost as frequently as cyclists.


Why? Because the target is \uline{~~6~~} . People complained that buses were late and infrequent. \uline{~~7~~} , the number of buses and bus lanes were increased, and drivers were \uline{~~8~~} or punished according to the time they took. And drivers hit these targets. But they \uline{~~9~~} hit cyclists. If the target was changed to \uline{~~10~~} , you would have more inspectors and more sensitive pricing. If the criterion changed to safety, you would get more \uline{~~11~~} drivers who obeyed traffic laws. But both these criteria would be at the expense of time.


There is another \uline{~~12~~} : people became immensely inventive in hitting targets. Have you \uline{~~13~~} that you can leave on a flight an hour late but still arrive on time? Tailwinds? Of course not! Airlines have simply changed the time a \uline{~~14~~} is meant to take. A one-hour flight is now billed as a two-hour flight.


The \uline{~~15~~} of the story is simple. Most jobs are multidimensional, with multiple criteria. Choose one criterion and you may well \uline{~~16~~} others. Everything can be done faster and made cheaper, but there is a \uline{~~17~~} . Setting targets can and does have unforeseen negative consequences.


This is not an argument against target-setting. But it is an argument for exploring consequences first. All good targets should have multiple criteria \uline{~~18~~} critical factors such as time, money, quality and customer feedback. The trick is not only to \uline{~~19~~} just one or even two dimensions of the objective, but also to understand how to help people better \uline{~~20~~} the objective.