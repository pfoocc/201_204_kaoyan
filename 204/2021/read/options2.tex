\item Some people argue that food self-sufficiency in the UK would \uline{~~~~}.
\begin{tasks}
	\task be hindered by its population growth
	\task contribute to the nation's well-being
	\task become a priority of the government
	\task pose a challenge to its farming industry
\end{tasks}
\item The report by the University of Leeds shows that in the UK \uline{~~~~}.
\begin{tasks}
	\task farmland has been inefficiently utilized
	\task factory-style production needs reforming
	\task most land is used for meat and dairy production
	\task more green fields will be converted for farming
\end{tasks}
\item Crop-growing in the UK is restricted due to \uline{~~~~}.
\begin{tasks}
	\task its farming technology
	\task its dietary tradition
	\task its natural conditions
	\task its commercial interests
\end{tasks}
\item It can be learned from the last paragraph that British people \uline{~~~~}.
\begin{tasks}
	\task rely largely on imports for fresh produce
	\task enjoy a steady rise in fruit consumption
	\task are seeking effective ways to cut calorie intake
	\task are trying to grow new varieties of grains
\end{tasks}
\item The author's attitude to food self-sufficiency in the UK is \uline{~~~~}.
\begin{tasks}
	\task defensive
	\task doubtful
	\task tolerant
	\task optimistic
\end{tasks}