We're fairly good at judging people based on first impressions, thin slices of experience ranging from a glimpse of a photo to five-minute interaction, and deliberation can be not only extraneous but intrusive. In one study of the ability she called``thin slicing,'' the late psychologist Nalini Ambady asked participants to watch silent 10-second video clips of professors and to rate the instructor's overall effectiveness. Their ratings correlated strongly with students' end-of-semester ratings. Another set of participants had to count backward from 1,000 by nines as they watched the clips, occupying their conscious working memory. Their ratings were just as accurate, demonstrating the intuitive nature of the social processing.


Critically, another group was asked to spend a minute writing down reasons for their judgment, before giving the rating. Accuracy dropped dramatically. Ambady suspected that deliberation focused them on vivid but misleading cues, such as certain gestures or utterances, rather than letting the complex interplay of subtle signals form a holistic impression. She found similar interference when participants watched 15-second clips of pairs of people and judged whether they were strangers, friends, or dating partners.


Other research shows we're better at detecting deception from thin slices when we rely on intuition instead of reflection. ``It's as if you're driving a stick shift,'' says Judith Hall, a psychologist at Northeastern University, ``and if you start thinking about it too much, you can't remember what you're doing. But if you go on automatic pilot,you're fine. Much of our social life is like that.''


Thinking too much can also harm our ability to form preferences. College students' ratings of strawberry jams and college courses aligned better with experts' opinions when the students weren't asked to analyze their rationale. And people made car-buying decisions that were both objectively better and more personally satisfying when asked to focus on their feelings rather than on details, but only if the decision was complex—when they had a lot of information to process.


Intuition's special powers are unleashed only in certain circumstances. In one study,participants completed a battery of eight tasks, including four that tapped reflective thinking (discerning rules, comprehending vocabulary) and four that tapped intuition and creativity (generating new products or figures of speech). Then they rated the degree to which they had used intuition (``gut feelings,'' `` hunches,'' ``my heart''). Use of their gut hurt their performance on the first four tasks, as expected, and helped them on the rest. Sometimes the heart is smarter than the head.