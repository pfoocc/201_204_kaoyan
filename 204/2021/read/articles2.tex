With the global population predicted to hit close to 10 billion by 2050, and forecasts that agricultural production in some regions will need to nearly double to keep pace, food security is increasingly making headlines. In the UK, it has become a big talking point recently too, for a rather particular reason: Brexit.


Brexit is seen by some as an opportunity to reverse a recent trend towards the UK importing food. The country produces only about 60 per cent of the food it eats, down from almost three-quarters in the late 1980s. A move back to self-sufficiency , the argument goes, would boost the farming industry, political sovereignty and even the nation's health. Sounds great — but how feasible is this vision?


According to a report on UK food production from the University of Leeds,UK,


85 per cent of the country's total land area is associated with meat and dairy production. That supplies 80 per cent of what is consumed, so even covering the whole country in livestock farms wouldn't allow us to cover all our meat and dairy needs.


There are many caveats to those figures,but they are still grave. To become much more self-sufficient, the UK would need to drastically reduce its consumption of animal foods, and probably also farm more intensively — meaning fewer green fields and more factory-style production.


But switching to a mainly plant-based diet wouldn't help. There is a good reason why the UK is dominated by animal husbandry: most of its terrain doesn't have the right soil or climate to grow crops on a commercial basis. Just 25 per cent of the country's land is suitable for crop-growing, most of which is already occupied by arable fields. Even if we converted all the suitable land to fields of fruit and veg — which would involve taking out all the nature reserves and removing thousands of people from their homes — we would achieve only a 30 per cent boost in crop production.


Just 23 per cent of the fruit and vegetables consumed in the UK are currently home-grown, so even with the most extreme measures we could meet only 30 per cent of our fresh produce needs. That is before we look for the space to grow the grains, sugars, seeds and oils that provide us with the vast bulk of our current calorie intake.


