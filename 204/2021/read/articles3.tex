When Microsoft bought task management app Wunderlist and mobile calendar Sunrise in 2015,it picked two newcomers that were attracting considerable buzz in Silicon Valley. Microsoft's own Office dominates the market for ``productivity'' software, but the start-ups represented a new wave of technology designed from the ground up for the smartphone world.


Both apps, however, were later scrapped, after Microsoft said it had used their best features in its own products. Their teams of engineers stayed on,making them two of the many `` acqui-hires''that the biggest companies have used to feed their great hunger for tech talent.


To Microsoft's critics,the fates of Wunderlist and Sunrise are examples of a remorseless drive by Big Tech to chew up any innovative companies that lie in their path. ``They bought the seedlings and closed them down,'' complained Paul Arnold, a partner at San Francisco-based Switch Ventures, putting an end to businesses that might one day turn into competitors. Microsoft declined to comment.


Like other start-up investors, Mr. Arnold's own business often depends on selling start-ups to larger tech companies, though he admits to mixed feelings about the result: ``I think these things are good for me, if I put my selfish hat on. But are they good for the American economy?I don't know.''


The US Federal Trade Commission says it wants to find the answer to that question. This week, it asked the five most valuable US tech companies for information about their many small acquisitions over the past decade. Although only a research project at this stage, the request has raised the prospect of regulators wading into early-stage tech markets that until now have been beyond their reach.


Given their combined market value of more than \$5.5 trillion, rifling through such small deals—many of them much less prominent than Wunderlist and Sunrise—might seem beside the point. Between them, the five biggest tech companies have spent an average of only \$3.4 billion a year on sub- \$ 1 billion acquisitions over the past five years—a drop in the ocean compared with their massive financial reserves, and the more than \$130 billion of venture capital that was invested in the US last year.


However, critics say the big companies use such deals to buy their most threatening potential competitors before their businesses have a chance to gain momentum, in some cases as part of a`` buy and kill''tactic to simply close them down.


