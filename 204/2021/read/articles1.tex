``Reskilling'' is something that sounds like a buzzword but is actually a requirement if we plan to have a future where a lot of would-be workers do not get left behind. We know we are moving into a period where the jobs in demand will change rapidly, as will the requirements of the jobs that remain. Research by the World Economic Forum finds that on average 42 per cent of the``core skills''within job roles will change by 2022. That is a very short timeline.


The question of who should pay for reskilling is a thorny one. For individual companies, the temptation is always to let go of workers whose skills are no longer in demand and replace them with those whose skills are. That does not always happen. AT\&T is often given as the gold standard of a company who decided to do a massive reskilling program rather than go with a fire-and-hire strategy. Other companies including Amazon and Disney had also pledged to create their own plans. When the skills mismatch is in the broader economy though, the focus usually turns to government to handle. Efforts in Canada and elsewhere have been arguably languid at best, and have given us a situation where we frequently hear of employers begging for workers, even at times and in regions where unemployment is high.


With the pandemic, unemployment is very high indeed. In February,at 3.5 per cent and 5.5 per cent respectively, unemployment rates in Canada and the United States were at generational lows and worker shortages were everywhere. As of May, those rates had spiked up to 13.3 per cent and 13.7 per cent,and although many worker shortages had disappeared, not all had done so. In the medical field, to take an obvious example, the pandemic meant that there were still clear shortages of doctors, nurses and other medical personnel.


Of course, it is not like you can take an unemployed waiter and train him to be a doctor in a few weeks, no matter who pays for it. But even if you cannot close that gap, maybe you can close others, and doing so would be to the benefit of all concerned. That seems to be the case in Sweden: When forced to furlough 90 per cent of their cabin staff, Scandinavian Airlines decided to start up a short retraining program that reskilled the laid-off workers to support hospital staff. The effort was a collective one and involved other companies as well as a Swedish university.


