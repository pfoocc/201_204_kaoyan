To combat the trap of putting a premium on being busy, Cal Newport, author of Deep Work: Rules for Focused Success in a Distracted World, recommends building a habit of ``deep work'' – the ability to focus without distraction.


There are a number of approaches to mastering the art of deep work – be it lengthy retreats dedicated to a specific task; developing a daily ritual; or taking a ``journalistic'' approach to seizing moments of deep work when you can throughout the day. Whichever approach, the key is to determine your length of focus time and stick to it.


Newport also recommends ``deep scheduling'' to combat constant interruptions and get more done in less time. ``At any given point, I should have deep work scheduled for roughly the next month. Once on the calendar, I protect this time like I would a doctor's appointment or important meeting'', he writes.


Another approach to getting more done in less time is to rethink how you prioritise your day – in particular how we craft our to-do lists. Tim Harford, author of Messy: The Power of Disorder to Transform Our Lives, points to a study in the early 1980s that divided undergraduates into two groups: some were advised to set out monthly goals and study activities; others were told to plan activities and goals in much more detail, day by day.


While the researchers assumed that the well-structured daily plans would be most effective when it came to the execution of tasks, they were wrong: the detailed daily plans demotivated students. Harford argues that inevitable distractions often render the daily to-do list ineffective, while leaving room for improvisation in such a list can reap the best results.


In order to make the most of our focus and energy, we also need to embrace downtime, or as Newport suggests, ``be lazy''.


``Idleness is not just a vacation, an indulgence or a vice; it is as indispensable to the brain as vitamin D is to the body…[idleness] is, paradoxically, necessary to getting any work done,'' he argues.


Srini Pillay, an assistant professor of psychiatry at Harvard Medical School, believes this counterintuitive link between downtime and productivity may be due to the way our brains operate. When our brains switch between being focused and unfocused on a task, they tend to be more efficient.


``What people don't realise is that in order to complete these tasks they need to use both the focus and unfocus circuits in their brain,'' says Pillay.